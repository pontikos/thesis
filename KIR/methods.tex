\subsection*{Subjects}

We analysed 12,106 individuals, 6,744 cases (age of onset less than 16) from the Genetic Resource Investigating Diabetes cohort,
and 5,362 controls from the British 1958 Birth Cohort (1958BC).
All subjects were of white European ancestry with written informed consent and Ethics Committee/Institutional Review Board approval.
The DNA for the cases and controls was prepared using the same protocols in Cambridge and in Bristol respectively.
All samples were cell-line derived.


\subsection{\emph{KIR3DL1/S1} Data}


\subsubsection*{qPCR Data}

A subset of 1,629 samples, 816 cases and 813 controls, were analysed with quantitative Polymerase Chain Reaction (qPCR) on the LightCycler 480 Real-Time PCR Instrument.
Primers were designed to determine copy numbers of \emph{KIR3DL1} and \emph{KIR3DS1}.
The housekeeping gene \emph{STAT6}, known to always be present in two copies, was used as a reference.
The forward/reverse primers and probe sequences for \emph{KIR3DL1}, \emph{KIR3DS1} and \emph{STAT6} are summarised in Table~S\ref{table:qPCR-primers}.
The qPCR assays were developed in collaboration with \citet{Jiang:2012cf}.

A subset of 1629 samples, 816 cases and 813 controls, were analysed
with qPCR on the LightCycler
480 Real-Time PCR Instrument.  Primers were designed to determine copy
numbers of \emph{KIR3DL1} and \emph{KIR3DS1}.  The gene
\emph{STAT6}, known to always be present in two copies, was used as a
reference.  The forward/reverse primers and probe sequences for
\emph{KIR3DL1}, \emph{KIR3DS1} and \emph{STAT6} are summarised in
Table~S\ref{Table-S2}.  The qPCR assays were developed in
collaboration with Jiang et al \cite{Jiang:2012cf}.

For each qPCR reaction, \SI{2}{\micro\litre} of DNA at
\SI{5}{\nano\gram\per\micro\litre} were used with \SI{5}{\micro\litre} of
Quantifast Multiplex PCR mastermix (\SI{0.25}{\micro\litre} primer mix,
\SI{0.045}{\micro\litre} probe mix and \SI{4.705}{\micro\litre} of water).
qPCR conditions were \SI{95}{\degreeCelsius} for \SI{5}{\minute}, followed by 40
cycles at \SI{95}{\degreeCelsius} for \SI{15}{\second} and
\SI{66}{\degreeCelsius} for \SI{50}{\second}.  Data was collected at
\SI{66}{\degreeCelsius}.  The samples were tagged with three different dyes,
Fam for \emph{KIR3DS1}, Cy5 for \emph{KIR3DL1} and DFO for \emph{STAT6}, and
amplified on eighteen 384-well plates.  On all plates, samples were replicated
across four wells.  So that each plate contained a maximum of 96 samples.  Four
calibrator samples of known \emph{KIR3DL1/3DS1} copy number and one water
sample were included on all but one plate.  Cases and controls were distributed
evenly across all plates.  Four plates were analysed in duplicate.


The experiment files exported from the LightCycler gave us the crossingpoint (Ct) value for each dye-DNA conjugate.
The Ct value is the number of amplification cycles required for the fluorescence of a dye in a given well to reach a certain detection threshold.
Hence, a higher Ct value implies a smaller starting quantity of DNA in the reaction well.
The Ct value also depends on the efficiency of the qPCR reaction.
%http://blog.mcbryan.co.uk/2013/06/qpcr-normalisation.html 
The expected reaction efficiency is 100 percent so that each PCR cycle leads to exactly one duplication of the DNA.
To measure the true reaction efficiency we can measure the increase in Ct when the concentration is diluted two-fold.
The slope is indicative of the reaction efficiency: 
if the decrease in Ct is larger than one than that indicates that the reaction is less than 100 percent efficient.

$ \textrm{Efficiency} = 2^{-\frac{1}{slope}} $

The Ct value also depends on the starting quantity of DNA which varies between wells.

To account for this a reference gene \emph{STAT6} is used, always present in two copies.
By subtracting from the Ct value of the reference dye-DNA conjugate, DFO-STAT6, we obtained the baseline relative $\Delta$Ct value for Fam-KIR3DL1 and Cy5-KIR3DS1.  
In theory negative values of $\Delta$Ct indicate two or more copies, and positive values, two copies or less, but the offset is not necessarily respected due to differences in dye efficiency.


As part of our quality control (QC), we dropped $64$ samples which did not yield a DFO-STAT6 Ct reading in all four well replicates
or for which the maximum difference in DFO-STAT Ct between wells was larger than 2.
All remaining samples were summarised by the $\Delta$Ct median of the four well replicates.


%\paragraph{Quality Control and Normalisation}

%The QC procedure for the controls is explained in more detail in \citet{Cooper:2012il}.
%Post QC, 6541 control samples were kept.

We normalised $\Delta$Ct values for \emph{KIR3DL1} and \emph{KIR3DS1} across plates by a linear transformation so that
the mean $\Delta$Ct of the two most distinguishable copy number groups, one and two copies, were aligned across all nineteen plates.
The group means were identified using the k-means algorithm.
%Furthermore samples for which the range in background Ct values was larger than $2$
Visual inspection of the data distributions by plate led us to drop plate 22 because it appeared excessively noisy (Figure~S\ref{figure:normalised-median} and Figure~S\ref{figure:QC}).


%The first method aligned the $\Delta$Ct values of calibrator samples containing known copy numbers, the second, aligned the two peaks of the $\Delta$Ct histogram corresponding to one and two copies.
%Of the two methods, the second peak-normalisation method was preferred as it allowed for normalisation of the leftovers plate which did not contain any calibrator samples and additionally yielded better repeatability.
After normalisation, samples repeated in different plates showed good reproducibility of the \emph{KIR3DL1} and \emph{KIR3DS1} $\Delta$Ct values (Figure~S\ref{figure:repeatability}).
Samples repeated across plates were summarised by the median of their repeated value.
%such as the calibrator samples 
Following QC, we were left with 1,474 individuals, 747 cases and 727 controls, for analysis (Figure~S\ref{figure:QC}).

%which is still the largest cohort to date for T1D association.
%A word of caution because the DNA quality in cases is poorer, excessive QC will eliminate cases rather than controls
%Dropping samples is a judgment call which introduces bias.
%\footnote{John, I still need to confirm this by looking at whether KIR genotypes agree.}

\subsection*{HLA typing}

In order to test for interaction with HLA-Bw4,
genotyping of HLA alleles was done using the DYNAL technologies from Invitrogen in a subset of 5,603 individuals,  2,922 cases and 2,681 controls (Table~S\ref{table:HLA-typing}).
%The HLA-DRB1, HLA-DQB1, HLA-A, HLA-B, and HLA-C
The HLA-A and HLA-B genes were typed at four-digit allele resolution using Dynal RELI SSO assays (Invitrogen, Paisley, U.K.).
The epitope classification of HLA-A and HLA-B alleles is given in Table~S\ref{table:HLA-epitope}.


\subsection{SNP data}

%\subsubsection*{Genome Wide SNP Array}
All 12,106 samples were genotyped using ImmunoChip, a custom Illumina 200K Infinium high-density SNP array \citep{immunochip},
according to the manufacturer's protocol, and processed at the University of Virginia in Charlottesville, USA.
%The dense SNP map for 186 confirmed and distinct loci, mainly derived from GWA studies, which have been associated (required P , 5 × 1028) with one or more autoimmune diseases,
%consisted of all known SNPs, and small insertion deletions, in the dbSNP database, in the 1000 Genomes project (February 2010 release)
%and in additional sequencing data provided by collaborators.
%The ImmunoChip contains 196,524 polymorphisms (718 insertion/deletions and 195 806 SNPs) \citep{immunochip}.
At each SNP, two readings were obtained, the Log R Ratio, indicative of copy number, and theta, indicative of allelic frequency.
On ImmunoChip there are 30 SNPs which fall in the \emph{KIR3DL1} region according to the human genome reference build36/hg18.
The \emph{KIR3DS1} gene is missing from build36/hg18.
%we should probably include more SNPs



\subsection{\emph{KIR3DL1/S1} Copy Number Calling}

In the qPCR dataset, samples which yielded one or less Ct reading for Fam-KIR3DL1 or Cy5-KIR3DS1, but all four Ct readings for the reference DFO-STAT6,
were assumed to contain zero copies of \emph{KIR3DL1} or \emph{KIR3DS1}.
For the remainder of the samples, we called genotypes using a mixture of bivariate Gaussian distributions allowing for eight \emph{KIR3DL1/S1} genotype groups:
three common groups of two copy numbers (major homozygote, heterozygote, minor homozygote) and five rarer groups of lower or higher copy numbers (Figure~\ref{figure:fuzzy-genotyping}).
The initial mixing and covariance parameters for the mixture of bivariate Gaussian distributions were calculated from the clusters returned by k-means
with centers set to the eight copy number groups.
%the initial mixing parameters could be the expected frequencies as derived from Trowsdale KIR haplotype paper
The parameters were then further refined using an EM algorithm \citep{mixtools}.
After fitting the mixture model each sample is assigned a posterior probability of belonging to each of the eight genotype groups which allows for uncertainty in genotype calling.
%These posterior probabilities were taken into account in downstream statistical analysis.

We increased our sample size with the SNP data,
by using the subset of samples, 747 cases and 727 controls, common between the qPCR and SNP datasets to train a k-nearest neighbour classifier
to predict \emph{KIR3DL1/S1} copy number from  SNP signals on ImmunoChip.
For the k-nearest neighbour classifier we chose $k=3$ as it minimised the leave one out cross validation error (Table~S\ref{table:knn-loo}).
The classification was achieved by using the 3 nearest neighbours on the Log R Ratio and Theta signals from 30 SNPs lying within the \emph{KIR3DL1} region (Table~S\ref{table:ImmunoChip-SNPs}) which were associated with \emph{KIR3LD1-KIR3DS1} copy number in individual linear regression.
Applying the nearest neigbour classifier independently on 10 multiply imputed data set generated from the qPCR posterior probabilities,
we estimated the mean misclassification error at $1.5\%$ (Table~S\ref{table:knn-misclassification}). 


%This allowed us to impute the \emph{KIR3DL1/S1} copy number in 6,744 cases and 5,362 controls.
%The misclassification error rate of our k-nearest neighbour predictor was in leave-one-out cross-validation (Table~\ref{table:knn-misclassification}).

%(6,004 cases and 4,663 controls).

%\begin{table}[ht]
%\begin{center}
%\end{center}
%\end{table}

%The three common genotype groups are close to Hardy Weinberg Equilibrium given that the respective frequencies of KIR3DL1 and KIR3DS1 is 8:2.

\subsection{Testing for Association of \emph{KIR3DL1/S1} Copy Number with T1D}

%Furthermore a Fisher test on the contingency table of genotype counts (Figure~\ref{figure:fuzzy-genotyping}) showed no significant association of the genotype groups on T1D status.
We tested for association of T1D with the predicted copy numbers from the qPCR and SNP datasets using logistic regression.
We allowed for copy number prediction uncertainty in the association test for qPCR data by doing multiple imputation (N=10)
with the \texttt{R mitools package} \citep{mitools} on the genotype group membership probabilities \citep{Cordell:2006da}. 
%To boost our power we also tested for T1D association with copy number in \emph{KIR3DL1} and \emph{KIR3DS1} separately.
%From the effect size and significance of the genotype groups we derived the odd ratios and confidence intervals.
We allowed for statistical interaction with HLA-Bw4 by repeating the association test
in the subsets of individuals carriers of the respective HLA ligands,
HLA-Bw4 ligand for \emph{KIR3DL1} and the HLA-Bw4-80I putative ligand for \emph{KIR3DS1}.
We further tested for interaction with a more powerful case-only $\chi^2$ test \citep{Yang:1999wk,Cordell:2009jb}.
%After not detecting any association with copy number groups,
%In order to increase the power of our test, we reduced the degrees of freedom by grouping copy number genotypes into presence/absence genotypes, 





