
Many papers have suggested that KIR genes along with their HLA ligands may associate with T1D but none have found independent association except 2DS5 in the Latvian population
but the gene was suspiciously underepresented in the controls.
The evidence for KIR association with T1D independently of HLA is weak only in the Latvian population where the controls are not in HWE.


There are two studies first in \citeyear{vanderSlik:2003gq} then in \citeyear{NikitinaZake:2004jv} which initially reported two KIR genes KIR3DS2 \citep{vanderSlik:2003gq,NikitinaZake:2004jv} and KIR2DS2 \citep{NikitinaZake:2004jv} to be associated with T1D independently of HLA class I in the Dutch and Latvian populations.
However both of these studies were underpowered (no more than hundred individuals each way) and none of reported p values were significant after correction for multiple testing.

\cite{Shastry:2008id} are still convinced that there is association in the Latvian population whereas \cite{vanderSlik:2007hi,Mehers:2011fj} have since moved to the model approach.

Independent association has not since been replicated in various other ethnicities Finnish \citep{Middleton:2006ba},  Japanese \citep{Mogami:2007gj}, Chinese \citep{Zhi:2011kl}, Bassque \citep{Santin:2006hh}.
Unsuprisingly when the HLA class I ligands are added to the analysis association is found as expected because HLA-B and HLA-C are associated with T1D.

The largest case-control study yet points to a SNP within the KIR2DL2 gene \citep{RamosLopez:2009jf}

\citet{Mehers:2011fj} claim that is the overall balance of inhibitory or activating KIRs, rather than the effects fo invidual receptors, that ultimately
has the greatest effects on NK cell function.
They therefore introduce multiple composite KIR-HLA models \citep{vanderSlik:2007hi} to take into account the cumulative effects of individual KIR-HLA class I
and II gene associations in children with T1D.
These models do however seems a little far-fetched.

In our study we have looked at KIR-3DS1/3DL1 because the corresponding HLA-B alles are highly associated independently of HLA Class II whereas the corresponding HLA-C ligands for KIR-2DL2 are not associated after controlling for HLA-B and HLA Class II \citep{Howson:2009bl}.

KIR-2DL2/2DS2 are part of the centromeric section of KIR whereas KIR-3DS1/3DL1 are on the telomeric section.
If there is independent association of KIR with T1D our analysis suggests that it is unlikely to lie with genes in high LD with KIR-3DS1/3DL1 on the telomeric section of KIR.

Why did we not find association with T1D?
The expression of KIR surface markers is variegated \citep{Pascal:2007wf,Li:2008ea}
NK cells only play a part in the early stages of T1D
also patients with long-standing T1D appear to have a lower expression of p30/p46 NK-activating receptor molecules compared with control subjects \citep{Rodacki:2007ht}
but this might be the consequence of the disease rather than the cause.
In the cases where T1D is the consequence of a $\beta$-cells infection by the Cosaxie virus \citep{Rodacki:2007ht}, then the effect of this KIR gene could be stronger.
KIR genes corelate with age of onset \citep{Mehers:2011fj}

Because T1D is the result of an over-active immmune system, activitating KIR genotypes would be expected to be more frequent in cases than in controls.





\begin{table}[h]
\begin{tabularx}{\textwidth}{lllll}
\rowcolor{Gray}
   & Study                       & Pop             & cases & controls \\
1  & \citet{vanderSlik:2003gq}   & Dutch           & 149  & 207 \\
2  & \citet{NikitinaZake:2004jv} & Latvian         & 98   & 100 \\
3  & \citet{Middleton:2006ba}    & Finnish         & 137  & 101 \\
4  & \citet{Santin:2006hh}       & Basque          & 76   & 71 \\
5  & \citet{PARK:2006km}         & South Korean    & 139  & 132 \\
6  & \citet{Mogami:2007gj}       & Japanese        & 204  & 240 \\
7  & \citet{vanderSlik:2007hi}   & Dutch           & 275  & 215 \\
8  & \citet{Shastry:2008id}      & Latvian         & 98   & 100 \\
9  & \citet{RamosLopez:2009jf}   & Belgian         & 394  & 401 \\
10 & \citet{RamosLopez:2009jf}   & German          & 380  & 315 \\
11 & \citet{Jobim:2010}          & South Brazilian & 248  & 250 \\
12 & \citet{Zhi:2011kl}          & Chinese Han     & 259  & 262 \\
13 & \citet{Mehers:2011fj}       & British         & 394  & 168 \\
14 & \citet{Pontikos:2014ho}     & British         & 6744 & 5362 \\
\end{tabularx}
\caption{
  \label{table:kir-t1d-studies}
  Fourteen known previous \gsl{KIR} studies in \gsl{T1D}.
}
\end{table}





\begin{table}[ht]\footnotesize
\begin{center}
\begin{tabular}{ccccccccccccc}
\hline
    & 2DS2 & 2DL2 & 2DL3 &  2DL1 & 3DL1 & 3DS1 & 2DL5 & 2DS3 & 2DS5 & 2DS1 & 2DS4 \\
\hline
  1 & +/+  &      &      &  -/-  & -/-  &      &      &      &      &      &      \\
  2 & +/+  & +/C1 & +/+  &       &      &      &      &      &      &      &      \\
  3 &      &      &      &       &      &      &      &      &      &      &      \\
  4 &      &      &      &       &      &      &      &      & (-)  &      &      \\
  5 &   -  &      &      &       &      &      &  -   &      &      &      &      \\
  6 &      &      &      &   -/- & +/+  &  -   &      &      &      &      &      \\
  7 &      & +/DR3&      &   -/- &      &      &      &      &      &      &      \\
  8 & -/+  & -/+  &  +/+ &       &      &      &      &      &      &      &      \\
\hline
\end{tabular}
\end{center}
\end{table}


%inhibiting
KIR2DL1
KIR2DL2
KIR2DL3
KIR2DL5
KIR3DL1
%activating
KIR2DS1
KIR2DS2
KIR2DS3
KIR2DS4
KIR2DS5
KIR3DS1

%1 vanderSlik:2003gq
KIR2DL1&94.6&97.6
KIR2DL2&55.7&48.4
KIR2DL3&91.9&92.3
KIR2DL5&50.3&46.9
KIR3DL1&96.0&96.1
KIR2DS1&36.2&35.7
KIR2DS2&55.7&47.8
KIR2DS3&24.8&27.1
KIR2DS4&40.9&42.0
KIR2DS5&32.9&27.1
KIR3DS1&38.9&33.3

%2 NikitinaZake:2004jv
KIR2DL1&95&98
KIR2DL2&81&32
KIR2DL3&86&91
KIR2DL5&65&55
KIR3DL1&92&94
KIR2DS1&43&27
KIR2DS2&53&25
KIR2DS3&35&19
KIR2DS4&94&92
KIR2DS5&29&22
KIR3DS1&40&27
KIR2DL4&98&100
KIR3DL2&98&100
KIR3DL3&98&100

%3 Middleton:2006ba
KIR2DL1&97.1&100
KIR2DL2&35.0&41.6
KIR2DL3&94.9&96
KIR2DL5&46.0&55.4
KIR3DL1&92.7&93.1
KIR3DS1&40.1&49.5
KIR2DS1&43.1&48.5
KIR2DS2&38.7&41.6
KIR2DS3&18.2&23.8
KIR2DS4&92.7&94.1

%4 Santin:2006hh
KIR2DL1&97&98
KIR2DL2&52&62
KIR2DL3&93&95
KIR2DL5&49&66
KIR3DL1&89&90
KIR2DS1&48&54
KIR2DS2&52&63
KIR2DS3&24&25
KIR2DS4&80&85
KIR2DS5&35&43
KIR3DS1&53&63
KIR2DP1&99&99
KIR3DP1&99&99

%5 PARK:2006km
KIR2DL1&99.3&100
KIR2DL2&46.0&34.8
KIR2DL3&98.6&98.5
KIR2DL4&97.8&97.7
KIR2DL5&42.4&84.1
KIR3DL1&96.4&96.2
KIR3DL2&97.8&98.5
KIR3DL3&96.4&99.2
KIR3DS1&36.0&37.1
KIR2DS1&33.8&43.9
KIR2DS2&20.1&47.0
KIR2DS3&10.1&9.8
KIR2DS4&96.4&96.2
KIR2DS5&22.3&33.3

%6 Mogami:2007
KIR2DL1&98.8&98.5  
KIR2DL2&15.4&13.7 
KIR2DL3&98.8&99.0
KIR2DL4&100 &100
KIR2DL5&40.8&34.8
KIR2DS1&40.8&35.8
KIR2DS2&15.4&13.7
KIR2DS3&9.6&9.8 
KIR2DS4&87.1&85.3
KIR2DS5&34.6&30.8
KIR3DL1&99.6&100
KIR3DL2&99.6&99.5
KIR3DL3&100&100
KIR3DS1&44.1&36.8

%7 vanderSlik:2007hi
KIR2DS1&41.1&35.8
KIR2DS2&53.5&48.8
KIR3DS1&40.4&34.0

%8 Shastry:2008id
KIR2DL2&82.65&32
KIR2DL5&66.32&55
KIR2DS1&43.87&27
KIR2DS2&54.08&25
KIR2DS3&35.71&19
KIR3DS1&40.81&27

%9 RamosLopez:2009jf

%10 RamosLopez:2009jf

%11 Jobim:2010
KIR2DL1&95.6&97.6
KIR2DL2&49.2&54.4
KIR2DL3&87.9&86.4
KIR2DL4&99.2&100.0
KIR2DL5&56.0&49.6
KIR3DL1&95.2&97.6
KIR3DL2&100.0&100.0
KIR3DL3&100.0&100.0
KIR2DS1&46.4&36.4
KIR2DS2&52.8&53.6
KIR2DS3&33.9&33.2
KIR2DS4&85.2&95.2
KIR3DS1&47.6&42.4
KIR2DP1&100.0&100.0
KIR2DS5&37.1&34.0

%12 Zhi:2011kl
KIR2DL1&93.82&96.56
KIR2DL2&28.19&32.44
KIR2DL3&98.46&99.62
KIR2DL5&40.15&42.37
KIR3DL1&93.05&95.42
KIR2DS1&37.84&37.40
KIR2DS2&28.96&30.92
KIR2DS3&11.58&12.98
KIR2DS4&92.66&93.13
KIR2DS5&27.03&27.10
KIR3DS1&36.29&35.50

%13 Mehers:2011fj 
KIR2DL1&97.9&100
KIR2DL2&52.8&53.6
KIR2DL3&94.2&90.5
KIR2DL5&54.1&56.5
KIR3DL1&96.4&95.2
KIR3DL3&100&100
KIR2DS2&52.8&53.6
KIR2DS1&42.9&41.7
KIR2DS3&30.5&33.3
KIR3DS1&44.9&44
KIR2DS4&95.7&94.6
KIR2DS5&34.5&34.5

%14 Pontikos:2011fj







