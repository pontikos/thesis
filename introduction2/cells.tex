%\subsection{Cells of the Immune System}

In the peripheral blood we distinguish two main groups of cells, red blood cells which bind oxygen and white blood cells known as leukocytes which are cells of the immune system.
%fight off pathogens.
Erythrocytes and thrombocytes are the two types of red blood cells and they constitute the majority of the cells in the peripheral blood (99.9\%).
The leukocytes, which include lymphocytes, monocytes and granulocytes are much rarer in the peripheral blood, as they only constitute the remaining 0.1\% \citep{Murphy:2007tl}.
%(put reference here).
Yet in this rare cell population there exists a huge diversity in physical characteristics and function.
It is this diversity that enables the versatility of the immune system in neutralising all kinds of pathogens (acquired and innate immunity),
its ability to distinguish them from endogenous cells (self tolerance),
%The purpose of the immune system is to distinguish self from non-self and neutralise any pathogens which enter the body.
and its capacity to adapt to better counter future infections (adaptive immunity).

An important type of leukocyte in the adaptive immune response are T lymphocytes also known as T cells.
%the cell-mediated immune response.
After having undergone central selection in the thymus,
T cells in the peripheral blood have an affinity for foreign antigens but are tolerant to self.
Initally these cells are in a naive state (naive T cells) until presented with an antigen,
at which point they mainly differentiate into effector T cells, capable of mounting an immediate response,
but also into longer-lived memory T cells, capable of mounting a stronger and faster response in the future thus
resulting in long lasting immunity againt this pathogen (acquired immunity).
In order to moderate the scale of the immune response and preserve self-tolerance, some T cells also have a regulatory function on the immune response
mediated using small signalling molecules known as cytokines (for example Interleukin 2).
These regulatory T cells are important in preventing auto-immunity.
%but too much suppression can lead to immune deficieny.
%The balance between auto-immunity and immune deficiency is known as homeostatis.

%However too much immune suppression makes the body vulnerable
%A breakdown of self-tolerance leads to autoimmunity.

%There is a fine balance between autoimmunity and immune-deficieny

%A naive T cell (Th0 cell) is a T cell that has differentiated in bone marrow,
%and successfully undergone the positive and negative processes of central selection in the thymus.
%A naive T cell is considered mature and unlike activated T cells or memory T cells it has not encountered its cognate antigen within the periphery.
%Memory T cells are a subset of infection fighting cells that have previously encountered and responded to their cognate antigen
%Such T cells can recognize foreign invaders, such as bacteria or viruses, as well as cancer cells.
%Memory T cells have become experienced by having encountered antigen during a prior infection, encounter with cancer, or previous vaccination.
%At a second encounter with the invader, memory T cells can reproduce to mount a faster and stronger immune response than the first time the immune system responded to the invader.
%This behaviour is utilized in T lymphocyte proliferation assays, which can reveal exposure to specific antigens.


