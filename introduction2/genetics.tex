\subsection*{Etiology and Diagnosis of Type 1 Diabetes}

\Gls{T1D} (OMIM:222100), also known as insulin dependent diabetes mellitus (diabetes - \foreignlanguage{greek}{diab\'hths},
a passer through, and mellitus - \foreignlanguage{greek}{m\'eli}, honey), is a disease reported as early as $1500$ BC \citep{Poretsky:2010wr}.
It holds its name from the characterisitic symptom of excessive discharge of high-glucose urine (glycosuria or hyperglycemia-induced osmotic diuresis polyuria).
It has since been established that this symptom is the consequence of persistently high levels of glucose (hyperglycemia) in the blood due to an insufficiency in insulin,
the hormone responsible for glucose regulation.
Long term high-glucose levels lead to dehydration, drowsiness, cardio-vascular complications, increased chances of morbidity and death.  
If left untreated T1D is a debilitating and life-threatening disease.

The cause of the insulin deficiency in T1D is an autoimmune reaction in which insulin and insulin-producing $\beta$-cells of the pancreatic islets
are progressively destroyed primarily through auto-reactive T cells \citep{Todd:2010bl}.

%, a hormone essential in regulating blood glucose levels, which causes the clinical symptoms associated with T1D:

In the last 50 years, the number of cases of T1D worldwide has increased and is predicted to continue increasing in the next decade,
affecting mainly children under the age of 5
\cite{Patterson:2009gj}.
%T1D manifests itself typically under the age of 20 which advocates a disease with a strong genetic predisposition.
The World Health Organization reported that in August 2011 around 34 million people worldwide were diagnosed with T1D.
%Glycaemia and diabetes are rising globally, driven both by population growth and ageing and by increasing age-specific prevalences.
%and affects around 5\% of the UK population \cite{Levy:2011wz}.

Presently there is no cure for T1D.
%T1D is diagnosed by screening for consistently high levels of glucose in the blood at which point
The only existing treatment is the regular intravenous administration of exogenous insulin.  
%This treatment is and there are efforts in developing more convenient method of delivering of insulin.
Early detection of T1D onset by
testing for presence of auto-antibodies against insulin and its precursors
%are strong indicators of T1D onset.
is therefore crucial to enable the development of therapies which can delay the onset, reduce the symptoms and potentially prevent the development of the disease.
%to restore normal glucose levels 
%Another line of research early detection is crucial for applying therapies 
One such therapy currently undergoing clinical trials in our lab attempts to restore immune tolerance with low-dose interleukin-2 (IL-2).
I will be presenting my preliminary analysis of these trials later in my thesis.
%Effective preventive interventions are needed, and health systems should prepare to detect and manage diabetes and its ramifications \cite{Danaei:2011hg}.


%(and hinder ketonacidocis)
%There are numerous genetic markers which increase susceptibility to T1D but these only indicate a pre-disposition to the disease.  
Because T1D is an auto-immune disease, there are quantitative features of the immune system, such as presence of insulin auto-antibodies, which can predict its onset.
%A more promising form of screening which further elucidate the molecular mechanisms which link the genetic factors to the T1D phenotype, is to study the immune phenotype.
But higher resolution study of the immune phenotypes, involving quantification of the populations and types of cells of the immune system,
promises to elucidate the molecular mechanisms of T1D which link genes, environment and phenotype.
Hopefully giving us a change to intervene.



\subsection{Heritability of Type 1 Diabetes}

%T1D is an auto-immune disease whereby cells of the immune system target and destroy the body’s own insulin secreting cells of the pancreas, the $\beta$-cells.

%Every trait is the result of a genetic and environmental influence and interaction.
Patterns of familial clustering suggest that some portion of T1D risk is genetically inheritable.
However since families also tend to share environment, it is difficult to decouple the genetic from the environmental effect.
%The genetic heritability of a trait independent of environment is known as the narrow sense heritability.
%Environment heterogeneity can lead to confounding and underestimation of heritability
One way of assessing the genetic influence relative to that of the environment is from the T1D concordance rate in \gls{MZ} and \gls{DZ} twins.
MZ twins have identical \gls{DNA} whereas DZ twins, like other siblings, share on average half of their DNA.
So if we suppose that the environmental effect is the same for both \gls{MZ} and \gls{DZ} twins, then the ratio of T1D concordance in \gls{MZ} twins ($\lambda_{MZ}$) 
over that in DZ twins ($\lambda_{DZ}$) should be indicative of genetic risk independently of environment \citep{Clayton:2009kf}:
$ \lambda_s = \frac{\lambda_{MZ}}{\lambda_{DZ}}$,
%An alternative measure of heritability is the ratio of incidences between monozygotic and dizygotic twins 

%\approx 5$

%The first measure of asssessing the genetic influence is the concordance rate.
%Under the age of 10 the concordance rate in MZ twins is good but after that age we have to wait longer for the second twin to also display symptoms.
\citet{Hyttinen:2003kn} estimated the pairwise concordance rate of $\lambda_{MZ}=42.9\%$ and $\lambda_{DZ}=7.4\%$ in 44 MZ and 182 DZ twins,
yielding $\lambda_s = \frac{42.9}{7.4} = 5.8$.
However, this could be an understimate since in the long-term follow-up study in 83 MZ twins, \citet{Redondo:2008} found that $\lambda_{MZ}=65\%$.
%and the auto-antibody concordance is $78\%$.  
This approach is extendable to relatives of type $R$ by considering the increase in risk $K_R$ in an individual when a relative of type $R$ has the disease compared to the population risk $K$.
%\citet{Spencer:2011be}
\citet{Risch:1987wm} defines this as the relative recurrence risk: $\lambda_R = \frac{K_R}{K}$.
One drawback of this approach is that more distant relatives tend to share less environmental factors which makes decoupling environment from genetics harder.
\cite{Risch:1987wm} estimated $K=.004$ and $K_R=.06$, yielding $\lambda_R=15$.
This confers a huge genetic risk but is likely to be an overestimate as the population risk is closer to $5\%$.

%The sibling recurrence risk is the probability that a sibling of an affected individual is also affected.
%The sibling relative risk which is the ratio of the sibling recurrence risk and the overall disease prevalence, is used by geneticists in planning and evaluating studies aimed at discovering genes conferring susceptibility to disease.

%$\lambda_s = \frac {Pr( Y_j=1 | Y_k=1)} {Pr( Y_k=1 )} $

%For T1D the population risk is about $5\%$ so $Pr( Y_k=1 ) = .05$


%The age of diagnosis concordance is highest in MZ twins when one develops T1D under the age of 10 \citep{Redondo:2008} which suggests as form of diabetes with a stronger genetic risk.
%Past the age of ten, \citet{Redondo:2008} find we can wait up to 43 years for the second MZ twin to also develop T1D,
%which suggests a form of T1D where the genetic risk is lower but instead more susceptible to environmental triggers.
%The early age of onset is a clue as to how strong the genetic effect in relation to the environment.

%Also the time between autoantibody positivity and T1D diagnosis (low insulin) increases with age.
%maybe this is because we have more functioning insulin circulating


%This shows that even against an identical genetic background, the concordance is not $100%$ which illustrates the growing influence of environment and gene-environment interaction in lower risk twins.
%However, within families, the disorder follows no clear mode of inheritance and is generally thought to result from the combined effects of multiple genes interacting with non- genetic factors (Risch 1987; Thomson et al. 1988; Rich 1990).




\subsection{Genetic Architecture of Type 1 Diabetes}

Having established that there is a considerable genetic predisposition to T1D, we are interested in identifying causal variants in our genetic code which might lead
some insights onto the mechanism of the disease.
%We are interesting in discovering how many genes have an effect


Linkage studies study based on the recombination of biallelic genetic markers in families affected by T1D, first mapped a genetic risk factor to the HLA Class 2 region on chromosome 6.

According to \citet{Alper200689}, in the $6\%$ of sibs of a patient which are concordant for T1D, $16\%$ are MHC identical.
However in concordant MZ twins, only $33$ to $42\%$ share the same variant.  This suggests that other genes are important.

As supported by \citet{Metcalfe:2001}, the insulin gene INS is also implicated.
\citep{Metcalfe:2001} found that within 40 MZ twin pairs concordant for T1D, $87.5\%$ carried the high risk INS variant (Hph I), compared to $59.5\%$ in 47 discordant MZ twins.

%\paragraph {Linkage Studies}

%If we take into account the cumulative genetic risk and assume an additive model of risk

However, these risk variants only explain a percentage of the heritability, so it is clear that T1D must involve smaller effect size risk variants that are only detectable at sample sizes larger than those achievable by linkage studies.


%\paragraph {GWAS}

\Glspl{GWAS} using the high density SNP array, GeneChip 500K Mapping Array Set (Affymetrix chip), have confirmed strong association of T1D within the Human Leukocyte Antigen (HLA) loci (chr6p21) as well as over 40 other loci including notably \emph{INS} (chr11p15), \emph{CTLA4} (chr2q33), \emph{PTPN22} (chr1p13), \emph{IL2RA} (chr10p15) and \emph{IFIH1} (chr2q24) \citep{Burton:2007hta,Barrett:2009jq} (\url{www.t1dbase.org}).


Within the HLA region, the strongest effect comes from HLA Class II loci, HLA-DRB1 ($P= 6.0 \times 10^{-17}$) and HLA-DQB1 ($P= 8.8 \times 10^{-13}$),
but there is also an independent effect from HLA Class I loci, HLA-A and HLA-B \citep{Howson:2009bl}.


%IL2RA codes for the alpha chain of the IL2 cytokine receptor (better known as CD25), are situated outside of the HLA region.
%Within the intronic (potentially regulatory) IL2RA region, three single nucleotide polymorphisms (SNPs) have shown to be significantly associated in case-control studies (Lowe et al., 2007; Smyth et al., 2008; Maier et al., 2009)


%IL2RA codes for the alpha chain of the IL2 cytokine receptor (better known as CD25), are situated outside of the HLA region.
%Within the intronic (potentially regulatory) IL2RA region, three single nucleotide polymorphisms (SNPs) have shown to be significantly associated in case-control studies (Lowe et al., 2007; Smyth et al., 2008; Maier et al., 2009)

Association was replicated and the signal was further fine mapped with the specialised SNP chip, Illumina Infinum 200K ImmunoChip, a custom ImmunoChip SNP array
of $195,806$ SNPs with dense coverage in immunologically important loci designed for deep-replication and fine-mapping of 12 common auto-immune diseases.
%such as the HLA region 
The strongest effect comes from HLA Class II loci, \emph{HLA-DQB1} and \emph{HLA-DRB1}, but there is also a second independent effect from HLA Class I loci, HLA-A and HLA-B alleles \citep{Todd:2010bl},
for example the HLA-Bw4/Bw6 epitope (P=$6.57 \times 10^{-6}$) \citep{Nejentsev:2007dv}.
The HLA Class II genes code for membrane-bound proteins which expose extra-cellular antigens to T cells,
%it is hypothesized that they may play a role in the insulin presentation pathway.  
whereas the HLA Class I, expose fragments of the cell's internal peptides to its surface for inspection by immune cells such as Natural Killer (NK) cells.

SNP arrays target common variants ($MAF > 5\%$)
We know that many regions of the genome have been neglected by SNP arrays as they are poorly mapped on the reference genome.
Such a region is KIR which will be an object of study in Chapter x.

%Recent estimates of ’narrow sense’ heritability from GWAS SNPs using mixed effects models have come much closer to the total heritability, and suggest that most of the missing heritability in complex traits is probably due to a multitude of SNPs contributing tiny effects below GWAS significance thresholds. Rare variants may also explain some of the heritability in families.


%\subsection*{The many faces of type 1 diabetes}

%The clinical diagnosis of T1D is insulin insufficiency.
But how much of the estimated heritability do the genetics explain?
The narrow sense heritability of the disease (not taking into account environmental heterogeneity) tells us that we might still be missing some genetic risk factors albeit at some very low odd ratios.
Using a linear mixed model and including all 500K SNPs from the WTCCC, \citet{Speed:2012hi} were able to explain $74\%$ heritability.
This suggests that alot of the missing genetic heritability can be attributed to many SNPs of small effect which do not reach genomewide significance but cumulatively explain an sizeable portion the variance.


%Nick Cooper
%Genome wide association studies have tagged moderate numbers of implicated SNPs in a long list of complex conditions.
%Most of these signals are quite weak and only detectable in large cohorts.
%Even once detected, combining these hits into a predictive model explains only a small amount of the variance in most phenotypes, even when a condition is known to be highly heritable through twin studies.
%This shortfall was coined as ’missing heritability’, with various suggested causes including:

%rare variants not captured in GWAS arrays,
%structural variations,
%polygenic small effects,
%and gene-gene and gene-environment interactions.
%The mixed effect method [29] is able to provide an overall variance estimate that is unbiased by sample size.

%This contrasts with standard linear models, which would be vastly skewed by the huge variable-to-case ratio of the problem space.
%The mixed-effects method has been applied to seven large Wellcome Trust Case Control Consortium (WTCCC) cohorts, which include T1D. T1D is estimated to have the largest genetic contribution of any of the complex diseases studied on this scale [30]. The initial implementations [31] of the mixed effects models suffered from bias due to the structure of LD. Speed et al [32] have improved upon the precision of these heritability estimates by modifying the matrix of relatedness between samples according to local LD, and have developed open source software to run these analyses efficiently (LDAK). Table 9 [32] below shows the weighted estimates of narrow sense heritability for 7 complex diseases.
%Using a mixed effects method I will seek to examine whether early diagnosis T1D is more or less heritable than late diagnosis T1D, and furthermore, when this variance is partitioned into chromosomes whether there is a distinction in which parts of the genome are involved at different onset-ages.



