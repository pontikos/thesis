

\subsection*{Heritability of type 1 diabetes}

%T1D is an auto-immune disease whereby cells of the immune system target and destroy the body’s own insulin secreting cells of the pancreas, the $\beta$-cells.
Type 1 diabetes (T1D) is an autoimmune disease in which insulin-producing $\beta$-cells of the pancreatic islets are progressively destroyed primarily through autoreactive T cells \citep{Todd:2010bl}.
This leads to an eventual deficiency in insulin, a hormone essential in regulating blood glucose levels, which causes the clinical symptoms associated with T1D:
dehydration, drowsiness, increased chances of morbidity and death.


Patterns of familial segregation and concordance in identical (monozygotic) twins suggest that some portion of T1D is genetically inheritable.
%However, within families, the disorder follows no clear mode of inheritance and is generally thought to result from the combined effects of multiple genes interacting with non- genetic factors (Risch 1987; Thomson et al. 1988; Rich 1990).

From twin studies we can therefore assess the influence of the environment against the genetics.

Monozygotic twins share identical DNA, dizygotic twins share on average 50 percent of their DNA.
Twins are often raised together in their infancy so usually share environment (more so than siblings).


Monozygotic twins of probands diagnosed with T1D tend to also develop T1D sooner if the proband is under the age of 16 \citep{Redondo:2008}.
Past that age, we can wait up to 43 years for the twin to also develop T1D.
The long-term follow-up study of \citet{Redondo:2008} indicates that concordance is $65\%$.
Also the time between autoantibody and T1D diagnosis increases with age.

%The sibling recurrence rate

Linkage analysis in affected siblings have mapped a genetic risk factor the HLA Class 2 region on chromosome 6.

But since this only explained a percentage of the heritability it has long been understood that T1D is a complex disease with many risk variants of small effect size.

Genome Wide Association studies using the high density SNP array, GeneChip 500K Mapping Array Set (Affymetrix chip),
have confirmed strong association of T1D within the Human Leukocyte Antigen (HLA) region chr6p21 as well as other loci such as IL2RA.


Genome wide association studies (GWAS) have confirmed strong association of T1D within the Human Leukocyte Antigen (HLA) loci (chr6p21)
as well as over 40 other loci including \emph{INS} (chr11p15),
\emph{CTLA4} (chr2q33), \emph{PTPN22} (chr1p13), \emph{IL2RA} (chr10p15) and \emph{IFIH1} (chr2q24) \citep{Burton:2007hta,Barrett:2009jq} (\url{www.t1dbase.org}).

Within the HLA region, the strongest effect comes from HLA Class II loci, HLA-DRB1 ($P= 6.0 \times 10^{-17}$) and HLA-DQB1 ($P= 8.8 \times 10^{-13}$),
but there is also an independent effect from HLA Class I loci, HLA-A and HLA-B \citep{Howson:2009bl}.


%IL2RA codes for the alpha chain of the IL2 cytokine receptor (better known as CD25), are situated outside of the HLA region.
%Within the intronic (potentially regulatory) IL2RA region, three single nucleotide polymorphisms (SNPs) have shown to be significantly associated in case-control studies (Lowe et al., 2007; Smyth et al., 2008; Maier et al., 2009)


%IL2RA codes for the alpha chain of the IL2 cytokine receptor (better known as CD25), are situated outside of the HLA region.
%Within the intronic (potentially regulatory) IL2RA region, three single nucleotide polymorphisms (SNPs) have shown to be significantly associated in case-control studies (Lowe et al., 2007; Smyth et al., 2008; Maier et al., 2009)

Association was further replicated and the signal was fine mapped with the specialised SNP chip, Illumina Infinum 200K ImmunoChip, a custom ImmunoChip SNP array
of $195,806$ SNPs with dense coverage in immunologically important loci designed for deep-replication and fine-mapping of 12 common auto-immune diseases.
%such as the HLA region 
The strongest effect comes from HLA Class II loci, \emph{HLA-DQB1} and \emph{HLA-DRB1}, but there is also a second independent effect from HLA Class I loci, HLA-A and HLA-B \citep{Todd:2010bl}, for example the HLA-Bw4/Bw6 epitope (P=$6.57 \times 10^{-6}$) \citep{Nejentsev:2007dv}.
The HLA Class II genes code for membrane-bound proteins which expose extra-cellular antigens to T cells,
%it is hypothesized that they may play a role in the insulin presentation pathway.  
whereas the HLA Class I, expose fragments of the cell's internal peptides to its surface for inspection by immune cells such as Natural Killer (NK) cells.

The narrow sense heritability of the disease (not taking into account environmental heterogeneity) tells us that we might still be missing some genetic risk factors
albeit at some very low odd ratios.

SNP arrays target common variants.
We know that many regions of the genome have been neglected by SNP arrays as they are poorly mapped on the refence genome.
Such a region is KIR which will be an object of study in Chapter x.

%\subsection*{The many faces of type 1 diabetes}


%The clinical diagnosis of T1D is insulin insufficiency.

Associating phenotypes to genetics.

Discrete phenotypes such as case-control status.

Quantitative traits such as percentage of cells, mean expression of surface protein determined as mean fluorescence intensity.

Certain of these traits might be correlated so independent association testing will inflate FP in the same way that testing for colocolisation with shared controls
exaggerates relatedness of diseases.


