\chapter{Discussion}

%\epigraph{ {\fontfamily{frc}\selectfont \foreignlanguage{french} { C'est le temps que tu as perdu pour ta rose qui rend ta rose importante. }} }
%Tu te jugeras donc toi-même, lui répondit le roi. C'est le plus difficile. Il est bien plus difficile de se juger soi-même que de juger autrui. Si tu réussis à bien te juger, c'est que tu es un véritable sage.
%\epigraph{ {\fontfamily{frc}\selectfont \foreignlanguage{french} { Il n'y a pas de citadelle indefendable, il n'y a que des citadelles mal defendues. }} }
\epigraph{
{\fontfamily{frc}\selectfont \foreignlanguage{french} {
L'avenir n'est jamais que du présent à mettre en ordre, tu n'as pas à le prévoir, mais à le rendre possible.
%Pour ce qui est de l'avenir, il ne s'agit pas de le prévoir mais de le rendre possible.
}}}

Throughout my thesis I have considered, normalisation and clustering methods for computationally analysing flow cytometry and genetic datasets in order to characterise the immune cell types and the genetic variants which could influence \gls{T1D} risk.
These methods have the potential to be more efficient, formal, consistent, objective and generally better at dealing with uncertainty than manual analysis.
Consequently they could lead to more powerful statistical association testing.
%In the case of flow cytometry data, automating gating can also provide the added benefit of formalising manual gating, which remains to this day a very subjective, ad-hoc, process.

%Ultimately the goal of normalisation and clustering is to increase power for statistical association testing.
%Decomposing the variance into tech
%This is challenging as most genetically inherited biological traits are complex, with a significant proportion of the variation explained by the cumulation and interaction of a large number of small genetic effects.
%a multitude of smalls effects which cumulatively explain
%Most effects in biological variation are small, they therefore require large sample sizes and

%The most studied factor which improves the statistical power to detect such effects is sample size, and we are witnessing ever-increasing sample sizes as consequence of decreasing genotyping and sequencing costs.
%This is because the standard deviation tends to shrink with the inverse of the sample size.
%However, there is still much work to be done on how reproducibility.

%Poor reproducibility is in part due to phenotype heterogeneity.
%The severity of the disease phenotype are known to differ depending on the individual and the stage of the disease.
%If the disease is diagnosed earlier then the associated phenotype may be recurrent auto-antibody positivity or if diagnosed late, high blood glucose.
%In many disease, case heterogeneity is a major confounder and one which needs addressing by collection of additional clinical covariates during recruitment, to enable subgroup differentiation \citep{Traylor:2013}.

%This binary case/control classification of disease phenotypes can be too coarse, especially for late onset diseases, motivating the study of better defined quantitative intermediate phenotypes, known as endophenotypes.
%It is expected that these should have stronger correlation with genotype, since they are closer to gene expression.
%In my thesis, I have studied the such as, CD25 protein expression on T cell subsets (\Cref{chapter:il2ra}), pSTAT5 \textit{ex vivo} response in \glspl{PBMC} (\Cref{chapter:il2}) or the relative cell subset frequency over parent cell population.

%These endophenotypes are more quantitative, especially with high-throughput technologies such as flow cytometry, in practice there are still factors which influence the accuracy at which these can be measured.
%The methods by which these endophenotypes are quantified have become extremely relevant and I will focus next on their application to flow cytometry.

%For example in \Cref{chapter:il2ra}, I illustrated how minor changes in the clustering can influence the relative frequency and MFIs of the cell populations under study and the strong effect of outliers on the results.
%Some other influential factors were, staining batch effects as seen in \Cref{chapter:il2}, or long-term instrument sensitivity as seen in \Cref{chapter:il2ra}.

% Linda: I'm not sure what you mean by "or the relative cell subset frequency over parent cell population."  Why aren't you summarizing the findings from your other chapters at this point in the introduction to you Discussion section?


%\section{ Pre-requisites }
\section{ Challenges in automation of flow gating }

The methods discussed in this thesis suggest that there are many scenarios in which flow cytometry analysis can be automated.
There are, however, a number of outstanding challenges, some technical, some methodological and some even philosophical, in applying these methods.

%Nonetheless, I will attempt here to give an overview of important practical and theoretical considerations when implementing and applying these methods, in the context of my thesis.

\subsection{Performance: compute time and memory usage}
While the applications that have been studied in this thesis do not require real-time analysis, there are still practical limitations on the amount of memory and compute time demands.
It is well known in computer science that these resources are interchangeable.
For clustering, methods which rely on global knowledge of the data, such as computation of the complete pairwise distance matrix, have a large initial memory footprint ( data matrix of size $\frac{N \times (N-1)}{2}$ ), but only require one pass of the data.
A solution to the clustering solution is reached within just a few computational steps, by for instance,
selecting a distance cut threshold on the dendrogram, built from hierarchical clustering, in order to define K clusters. 
On the other hand, methods like K-means necessitate a much smaller memory footprint since they only compute the distance of every point to the K cluster means ( data matrix of size $N \times K$, where typically $K << N$), however, several updates of the matrix are required until the cluster centers are fixed.
The choice of which method to apply is very much dependent on the dataset.
For ungated flow cytometry, data matrices  contain over a million rows, so the complete pairwise distance matrix is too large to fit in memory.
I found that for the default R implementation, the maximum allocatable vector size is 1672.4 Gb, hence the distance matrix computation was only feasible in subsets of the order of $10000$, such as the CD4\positive lymphocyte subset, or data downsampled using the SPADE or RPART approaches in \Cref{chapter:il2}.
The data downsampling performed by SPADE, relies on first estimating the local density at each point and then preferentially thinning the data in regions of high density in order to even out the density across the whole sample.
The local density estimation step is itself computationally intensive since it needs to consider the distance from a single point to all other points in the sample to find the number of points lying within a certain radius.
I found this step could be greatly sped up with \acrfull{ANN} which uses the K dimensional trees (KD tree) lookup method.
KD-trees are a space-partitioning data structure for organising points in a k-dimensional space, making for an efficient way of storing a high-dimensional dataset to lookup proximal datapoints.
In theory, this approach could also be used to reduce the number of datapoints considered when applying mixture models to large datasets \citep{McLachlan:2004uw}.

\subsection{Consistency}
Consistency can be defined as the variation in a method's output in relation to its input.
If a clustering algorithm is consistent, one would expect that small perturbations to the data would lead to small changes in the clustering outcome.
However, algorithms which rely on initialisation using random starting positions, such as K-means, can reach different clustering solutions, even when rerun on the same data.
To guard against this when running K-means, initial cluster centers can be specified or the algorithm can be run with multiple restarts in order to increase the chances of finding a globally optimal solution.
As I will discuss later, a relatively small number of outlier events which occur commonly within a flow cytometry experiment, due to cells clumping together or debris, can also be detrimental to consistency; it follows that automated gating algorithms need to be robust to these.

\subsection{Accuracy}
When labelled data is available, the accuracy of a method is defined based on how frequently a method assigns the correct label.
Estimating accuracy relies on the existence of a test dataset, typically labelled using manual analysis or some other method.
In \Cref{chapter:il2ra}, I assessed accuracy by comparing the cluster proportions and means with those obtained from manual gating.
%In \Cref{chapter:il2}, the accuracy was determined in terms of RSS of the pSTAT5 response.
In \Cref{chapter:kir}, I used qPCR labelled data to assess the prediction accuracy of the \gls{knn} classifier.
However, labelled data may not always be available, especially in the case of flow cytometry data.
In addition, in the context of flow cytometry, even when labelled data is available, this approach may not always be ideal,
as it is merely comparing the relative agreement between methods rather than the objective truth.
A more useful alternative may be instead to assess the prediction accuracy with clinical outcome, case-control status or genotype, or to maximise reproducibility in repeated independent samples from the same individual under the same conditions.
%This is usually implicitly measured when doing an association test.
%The consistency-accuracy tradeoff is analogous to the variance-bias trade-off in statistics.
%Consistency and accuracy are typically measured on simulated data.

\subsection{Interpretability}
While an automated method might be accurate and consistent, it may be difficult to interpret the results.
%, much like a black box.
%In biology, we are often more interested in the process behind the result than in the result itself.
Improving the accuracy of a model by adding parameters can obfuscate the relationship between the input parameters and the clustering output.
\Acrfull{RF} and neural networks are examples of methods from which it is difficult to extract an interpretable model to justify the result.
This is an issue because
%generally people (and especially inquisitive minds like scientists) are not comfortable with applying methods they don’t understand.
as part of the iterative process of knowledge discovery, it is important to understand which combination of features make objects distinguishable.
%Since fully automated methods are hard to assess.
%Sometimes additional information which is available from previous studies can be used to guide the clustering.


%\paragraph{Parallelisation}
%Provided the data can be split into independent chunks, then this feature can be exploited with parallelisation.
%By make astute use of parallelisation, one can marry local and global methods to make efficient use of the computational facilities
%available such as compute farms of many nodes with limited memory and processing power.
%In such a scenario, each compute node can work on subset of the data to compute intermediate results which can then be combined.
%In the context of flow cytometry however, samples were not split but analysed individually on each compute node.
%In the future, it may be possible to exploit the hierarchical nature of the gating to conduct parallel analysis of independent subsets.

%\section{Clustering with prior knowledge}
%\section{Challenges of computational methods}
%\section{Dealing with noise using prior knowledge}

%Having gone through some of the theoretical and practical properties of computational clustering methods, I will now address challenges which are more pertinent to the data on which these methods are applied.
%%One compromise is that by adding more markers to delineate cell populations we are increasing the number of potential clusters and thus the tail of distribution of cluster sizes
%Variation due to batch effects in the preparation or sample ascertainment,
%or insufficient number of flow cytometric parameters


%Futhermore another concern with clustering and statistics in general is distinguising rare subsets from outliers.

%Individual outlying datapoints are harder to spot in smaller sample sizes and we are also generally more reluctant to exclude samples.
%In \Cref{chapter:kir}, a whole qPCR plate was excluded because the $\Delta$Ct distribution didn’t align well even after normalisation.
%Later I found that qPCR samples could have also been scored by considering their background Ct median and spread.
%This score would have been used to downweight these samples in the mixture model clustering which was used to assign them to copy number groups.  
%However in the case of systematic error, dropping individual samples could introduce bias.

%Another good means of spotting outliers, is visualisation by for example plotting summary statistics from multiple samples.
%Boxplots are the typical visualisation tool for viewing univariate intensity data but in flow cytometry where we typically have multivariate data, \gls{SPICE} presents an overview using pie charts of multiple samples \citep{Roederer:2011hy}.
%%An alternative visual representation are radarplots

%These types of scores help to capture the uncertainty in calling outliers.
%While certain outliers can be confidently called independently because their values lie at the extremes of the range of the instrument, others are less clear and should not be discarded completely because in light of new data they may be assigned to existing clusters or to new clusters.



%\section{Advantages and disadvantages of top-down clustering}
%
%Sometimes prior knowledge can be used to build a top-down tree-like approach to clustering, such as manual gating in flow cytometry as described in \Cref{chapter:il2ra} and \Cref{chapter:il2}.
%This approach is advantageous from a practical perspective, it is efficient because the dataset is reduced as points are filtered at each step, and as not all dimensions are considered at the same time, simpler univariate and bivariate clustering algorithms which use density can be applied.
%%Also only considering a subset of markers at a time is easier to interpret.
%It also conforms to the established hierarchical model of cell lineages which is deeply engrained in the minds of certain immunologists.
%%Methods which use a binary structure are easy to interpret as they conform to the manual gating strategy which provides rules on how to obtain the leaf subsets.
%%Reduces ambiguity.
%On the other hand, a disadvantage is that this approach can easily miss important cell populations.
%Furthermore, the construction of trees tends to very sensitive to small perturbations in the underlying data and errors propagate at each step.
%These inconsistencies become more apparent the deeper the hierarchy.
%Errors could be mitigated by introducing conditional probabilities at each step but this would come at the cost of conserving all datapoints.
%Another shortcoming of this top-down approach is that it imposes a directionality on cell subsets which may not accurately reflect the biology.
%It also imposes an ordering on the influence of the markers on the gating which may not be the most appropriate for every dataset.
%%It may be more ambiguous markers are gated on first.
%Although this hierarchical approach does have advantages, I would advocate also trying more unsupervised approach when sufficient data is available, for example \gls{RF}, to get some idea of the relative importance of the markers.
%%in the same way that phylogenies change as new animal species are discovered, cell lineages should also be influenced by the discovery of new cell subsets.
%%However, if no restriction are applied the rules can become over-complicated.
%%When a hierarchical gating approach is taken, errors in earlier steps propagate.
%%This approach is chosen for manual gating as it agrees with our very linear way of thinking.
%%This complicates automated approaches since errors which occur in earlier steps of the gating are not recoverable from.
%

%\section{ Distinguishing between outliers and rare subsets }
%\section{Prioritising normalisation or clustering}
%\section{To move data or to move gate: normalisation vs clustering}

%Sometimes it is better not to normalise and normalisation may hide the true biological effect.
%The question really boils down to whether we should normalise the samples or cluster each sample separately.
%Normalisation is effectively analogous to meta-clustering since we are attempting to match cell populations across samples.  
%Normalisation facilitates matching clusters across datasets but in doing can remove meaningful biological variation.  
%Theoretically if the data was perfectly aligned across days then gates should not be moved.  
%Threshold gates should not move unless they are relative to some internal control.
%The cell phenotypes which are usually measured are the mean intensity of a particular channel, or the percentage of cells.
%Depending on the shape of the distribution these can be more or less sensitive to the position of the gate.
%Normalisation removes unwanted experimental variation to make data comparable even when collected on different days,
%proccessed with different protocols or analysed with different instruments.
%However distinguishing between what is unwanted and what is meaningful variation relies ultimately on a bias-prone judgement call.
%As we have seen, different normalisation approaches make different assumptions about what is unwanted variation and the shape of the data.
%The actual choice of normalisation depends on the characteristics of the data we wish to compare.
%We have only considered the univariate density function here but identification can be extended to the multivariate case.
%Next let's consider normalisation of multivariate distributions.  
%The purpose of principal component correction is to remove unwanted correlation between samples.  
%When applying a linear transform to multivariate data the correlation between the dimensions is preserved since the multiplicative factors cancel out.
%However the covariance changes.
%Care must be taken when applying a non-linear transform.  
%The samples are related and we wish to exploit this structure without forcing data to be perfectly comparable.
%A less biased approach is to allow for different cluster location and shapes across datasets and instead incorporate the clustering in some sort of hierarchical framework.  
%Normalisation or meta-clustering is a necessary step when data is pooled across studies.  
%One issue with normalisation based on peak alignment is that the level of there is more noise variation than biological variation between samples so that in certain samples,
%so much so that in certain samples the peaks are only identifiable after preliminary gating of lymphocytes.  

%\paragraph{Normalising after clustering: good for matching cluster across samples}
%
%In the case when the centre of mean of the gate has moved, after a few E iterations of the EM algorithm for example, it possible to realign the gates on the data.
%The advantage of this approach is that it doesn’t require all the data to be moved.
%But normalisation is still required in the meta-clustering step to compare properties of the clusters.
%Normalisation is effectively analogous to meta-clustering since we are attempting to match cell populations across samples.  
%Normalisation facilitates matching clusters across datasets but in doing so can remove meaningful biological variation.  
%
%The samples are related and we wish to exploit this structure without forcing data to be perfectly comparable.
%A less biased approach is to allow for different cluster location and shapes across datasets and instead incorporate the clustering in some sort of hierarchical framework.  
%Normalisation or meta-clustering is a necessary step when data is pooled across studies.
%
%\paragraph{Normalising before clustering: good for pooling to find identify rare populations}
%
%If applied before the clustering it enables data pooling to identify clusters across all samples.
%Pooling may improve the coefficient of variation of populations but also facilitates
%matching clusters across samples, a step known as meta-clustering.
%Pooling is crucial for identifying rare groups which are difficult to identify within one sample.
%If the alignment isn’t perfect, the clustering results can be refined across all samples.
%
%Theoretically, if flow cytometry data was perfectly aligned across days then gates should not be moved, unless they are dependent on some internal control.  
%For example a threshold gate on a positive subset may be defined in relation to another population of cells.
%
%As discussed in \Cref{chapter:il2ra}, the cell phenotypes usually measure, the mean intensity on a particular marker, or the percentage of cells can be more or less sensitive to the position of the gate, depending on the shape of the distribution.

%\paragraph{Some issues with normalisation}
%
%Normalisation removes unwanted experimental variation to make data comparable even when collected on different days, processed with different protocols or analysed with different instruments.
%However distinguishing between what is unwanted and what is meaningful variation relies ultimately on a bias-prone judgement call.
%
%As we have seen, different normalisation approaches make different assumptions about what is unwanted variation and the shape of the data.
%
%The choice normalisation depends on the characteristics of the data we wish to compare.
%
%The Bayesian question of relative weight of prior vs data is very relevant to gating.
%%The initial position of the gates are the priors.
%In \Cref{chapter:il2ra} the question of whether it better to have an absolute threshold or one relative to the data, showed that keeping the same threshold for all samples analysed on the same day lead to better reproducibility than the per sample user-set threshold.
 %
%One issue with normalisation based on peak alignment is that the noise variation is greater than the biological variation between samples, so much so that in certain samples, the peaks are only identifiable after preliminary gating of lymphocytes.
%
%We have only considered the univariate density function here but identification can be extended to the multivariate case.
%
%\paragraph{Normalisation of multivariate distributions}
%
%%The purpose of principal component correction is to remove unwanted correlation which may exist between samples or GC content.  
%When applying a linear transform to multivariate data the correlation between the dimensions is preserved since the multiplicative factors cancel out.
%However the covariance changes.
%Applying a non-linear transform also changes the correlation.
%If the data is binned using flowBin or clustered using SPADE then a transform could be a mapping like Earth Moving Distance to make both distributions of event proportions identifical across samples.
%
%\paragraph{Combining transforms with clustering}
%
%Choosing an optimal transform in flow data is not trivial.
%Transforms tend to be channel specific and sometimes even sometimes sample specific.
%%Per sample transform required like flowClust but using the sliding window approach.
%%Some prior knowledge is required to know what makes a sensible density plot.
%%Care needs to be taken not to introduce spurious peaks.
%The \Rpackage{flowClust} proposes estimating the Box-Cox power transform exponent as part of the clustering using a numeric optimiser.
%However the parameters need to be transformed back, for allowing for a different transform per sample.
%This is an issue if comparing intensity data but obviously if comparing cell ratios then the transform is of little consequence.
%Nonetheless the transform can have an important effect on the clustering result.
%
%Clustering is an iterative method of discovery not a fully automated process.
%%Some tuning is required
%It will often be necessary to modify data preprocessing and model parameters until the result achieves the desired properties.  
%Clustering can also be viewed as a latent variable problem where the cluster labels are considered to be the missing data.
%
%\paragraph{Conclusion}
%
%Whilst normalisation is generally a necessary process for pooling data or comparing across samples, over-correction can be detrimental to the analysis.
%Normalisation is meant to be a simpler preliminary step to simply the clustering,
%however, in flow cytometry, because of batch effects and unequal number of events per sample, normalisation can be as challenging as performing the gating.
%%generally relies makes big assumptions and leads to major changes in the data.
%
%When there are insufficient data points, pooling is necessary.
%Clustering sometimes relies on pooled samples when they are not sufficient data points
%to form clusters such as in the case of rare subsets like 3-0 in KIR (\Cref{chapter:kir}) or Tregs in flow cytometry data.
%In this situation some form of normalisation is usually necessary.
%Unfortunately, multivariate normalisation is not always trivial and relies on defining linear transforms. 


%\section{ future application of methods }

\subsection{Choice of transformation}

Flow cytometry, and fluorescence data in general, tend to be highly positively skewed.
This is problematic because most clustering algorithms assume constant variance across the data range.
%rely on on variance decomposition.
While the skewness can be reduced by the means of tranforms based on the logarithm function, care needs to be taken as these influence the modality of lower intensity populations, especially those which overlap into the negative range, as illustrated in \Cref{figure:logicle-transform-w}.
Certain flow cytometry packages, such as the \Rpackage{curvHDR} erroneously apply an arcinsh transform, however such a function introduces a split in the data density around zero, giving rise to spurious cell populations.
In FlowJo, the transform can be customised visually by the manual gater, given the knowledge of what cell populations to expect.
The only existing automated  methods of optimally selecting a transform that I am aware of are the \texttt{R} packages \textsf{flowTrans} and \textsf{flowClust} \citep{flowTrans,flowClust}.
\textsf{flowTrans} assumes an underlying Gaussian distribution and uses \gls{ML} to estimate the optimal transform parameter.
\textsf{flowClust} applies a Box-Cox transform for which the lambda parameter, the exponent of the transform, is estimated using \gls{ML}.
Nonetheless, as I discussed in \Cref{chapter:intro} and showed in \Cref{figure:logicle-transform-w}, it is not always clear what transform to apply and, I would argue, aiming for a Gaussian distribution is a suboptimal criterion given the multimodality of the data.
When the number of populations is unknown, one approach could be to estimate the transformation parameter as part of the clustering process.


\subsection{Visualisation of higher dimensional datasets}

Visual inspection is a fundamental tool for quality control, discovery of data features like skewness or symmetry, looking for patterns, gene lists appearing in pathways, confirming clustering results, spotting outliers.
While visualisation works well for up to three dimensional data, information is lost when higher dimensional datasets are decomposed into a series of two-dimensional projections.
Clusters which exist in higher dimensions do not necessarily map to clusters in two dimensions.

Open repositories such as Cytobank\footnote{\url{https://www.cytobank.org/}} and FlowRepository\footnote{\url{http://flowrepository.org/}} \citep{Spidlen:2012hk} are encouraging researchers to share their annotated flow cytometry experiments along with their publications.
By combining data across experiments, both the number of samples and parameters measured are increasing, although care must be taken to avoid confounding by batch effects.
Also an upcoming biotechnology, time of flight cytometry (CyTOF), which combines mass spectometry with cytometry, will push the number of markers which can measured by experiment up to $34$ and potentially higher.
%, although the throughput is not as high as fluorescence flow cytometry.
%It cannot be used for sorting as the cells are destroyed when measured.
%Also, as it does not report side and forward scatter, live/dead markers are used instead to spot debris.
At the Gary Nolan lab in Stanford, and the Cancer Research Institute in the UK, mass cytometry is being adopted to analyse cell heterogeneity in cancer.
The analysis of datasets generated by this technology benefits greatly from the multidimensional visualisation techniques such as \gls{SPADE} \citep{Simonds:2011jh} and \gls{viSNE} \citep{Amir:2013jp}.
%These high-dimensional datasets require some form of low-dimensional visualisation and Dana Pe'er group at Columbia University has devised various tools to do so such as viSNE \citep{Amir:2013jp}.
%Clusters are not identifiable in two dimensions, in fact for clusters which are thus ignoring a large number of points.
%In fact K points are only garanteed to be separable in at least K+1 dimensions.
%An upperbound on the number of clusters in the data is the product of the univariate modes,
%whereas a lower bound is the maximum number of modes in one dimension.
%This has motivated research into how best to visualise high-dimensional data in two-dimensions with minimal loss of information.
In \Cref{chapter:il2}, I presented one these approaches, \gls{SPADE}, which relies on a network visualisation of a dataset using a minimum spanning tree.
\gls{viSNE} is a more probabilistic approach which uses stochastic neighbour embedding.
Visualising high-dimensional data using network representations is only informative if the number of datapoints is relatively small.
In flow cytometry, some clustering or binning of the data is applied to reduce the number of points.  
%There are many more which take more probabilistic approaches like stochastic neighbour embedding
%My opinion is that this fascination for visualisation boils down to a distrust of computer vision
%visualisation can also be misleading and lead to oversimplifications
%An important part of our job is to unearth trends or make them visually obvious.

\subsection{Ascertaining the number of clusters}


Certain transformations can facilitate the clustering task, however the challenge remains to determine the number of clusters, K, in a particular dataset.
This is an unsolved statistical problem, although a variety of approaches exist.

In univariate data, a sliding window approach can be used to estimate the number of modes/peaks.
The number of peaks returned is influenced by the span of the sliding window.
A large window span will tend to oversmooth the data, leading to fewer peaks while with a smaller window the number peaks called will increase, but so will the chances of picking up spurious peaks.
However the exact relationship between the window span parameter and the number of clusters returned is data dependent.

%Over the course of my PhD, I have spent a great deal of time assessing whether clustering can be fully automated, i.e made truly non-parametric.
One approach to estimating K is to start with an upper bound for K and then merge clusters together, as can be done with \Rpackage{flowMerge}.
Another approach would be to select a K that gave consistent clustering results across samples.
However, finding such a K across clusters is not always possible due to sampling variation.
In particular, rare clusters may only be visible when a sample is sufficiently large, and so may not be consistently identifiable across all samples.
%In cases when certain characteristics of the data are known, these can assist in the clustering especially with regards to identifying rarer subsets which 
In genetics for example, when genotyping low minor allele frequency variants, large samples sizes are required for homozygous individuals to be included.
%When distinguishing between rare subsets and noise can be hard.
In order to distinguish such rare subsets from noise, strong supporting evidence is required.
As illustrated in \Cref{chapter:kir} on qPCR data, prior evidence from \citet{Jiang:2012cf} supports the existence of the rare 3-0 \gene{KIR3DS1}-\gene{KIR3DL1} copy number group at that sample size.
In hindsight, the expected copy number group frequencies obtained from \citet{Jiang:2012cf} could have been used as prior group frequencies for all copy number groups in the clustering of the qPCR data.
The prior information obtained from the qPCR dataset, was then used as training data in the next step of the analysis, in order to identify clusters predictive of \gene{KIR3DS1}-\gene{KIR3DL1} copy number in the SNP dataset.
%The \gls{knn} algorithm was used to classify unlabelled points, for which qPCR data was unavailable, from the vote of their K nearest labelled neighbours.
%The optimal value of K was selected by minimising the \gls{LOOCV}.

However, prior evidence yields information about a cluster's expected proportion but little about its exact position and shape, which are usually experiment specific.
This can be due to the reliability of the instrumentation and sample preparation, but also due to the stability of the biological sample; DNA variation changes can be seen over many human generations, while variations in cell protein expression and cell populations fluctuate at a much shorter time scale.
%Supervised clustering can be really beneficial in the identification of rare subsets, since they may not always be visible in all samples.

Generally, in order for clusters to emerge, the number of events collected should be increased when possible, but this may also increase the within sample noise.


\subsection{ Within sample noise }

In flow cytometry, scatter channels for example, include many spurious events due to debris and cells clumping together creating doublets.
In fact, there are many technical and biological artefacts, in flow cytometry, which can lead to spurious clusters or outliers.
This is a problem because any clustering method or statistical test which relies on estimating a mean is potentially sensitive to outliers.
%An outstanding problem with supervised clustering for which I have not been able to find any satisfactory solution, is how to account for unlabelled data, or in the context of flow cytometry, nuisance clusters which are not included in the gating strategy.

The approach taken in top-down hierarchical manual gating is to filter out these datapoints.
Automatic filtering of outliers usually relies on density estimation in order to exclude low density points from belonging to any cluster, as implemented in methods such as SPADE \citep{Simonds:2011jh}.
%and consequently all points can influence the gating.
%However there are methods consider all points in the dataset.

An alternative solution, which does not rely on filtering, is to define, one or more background clusters to account for data points which are not part of the study.
Certain model-based methods do this already by defining a background mixture with a covariance defined on the entire sample which essentially "mops up" all points which are not assigned with high posterior weight to any of the known components.
The \Rpackage{mclust}, allows one to define a noise component with an expected frequency.

%\section{ Identifying outliers } 
%While certain outliers may be clearly identifiable because they lie at the extreme of the instrument range, far from the rest of the data, other may be found with intermediate values and so are harder to spot.
%Very large intensity values are usually considered as noise, but low intensity values are harded to call with certainty.
% and require large samples sizes to confidently call as outliers.
%Others which lie within a more plausible range of values 
%There is a soft approach of dealing with outliers and a hard approach.

%A sample from a single flow cytometry experiment contains millions of events, many of which are discarded as part of the manual gating process.
%Some of these are real biological clusters but are ignored because they are not part of the cell populations under study:
%for example the first gate drawn is usually on the lymphocytes whereas the monocytes and granulocytes are often ignored.
%This is in part because they are no markers included in the experiment defined on these cell populations which would allow any further division.
%Other events are discarded because they are outliers.

Another approach to filtering outliers, as adopted by the \Rpackage{flowClust}, is to downweight the effect of outliers on the parameter estimation of the mixture model by defining an "outlyingness" parameter which is inversely proportional to the Mahalanobis distance from a point to a cluster centre.  
Since the Mahalanobis distance from a point to a group of points is scaled by the covariance matrix, the distance is smaller to wider clusters than to tighter clusters:
\begin{equation} \label{equation:Mahalanobis}
(\boldsymbol{x_i}-\boldsymbol{\mu})^{\top}\boldsymbol{\Sigma}^{-1}(\boldsymbol{x_i}-\boldsymbol{\mu})
\end{equation}
where $\boldsymbol{x_i}$ is the datapoint, $\boldsymbol{\mu}$ is the cluster mean and $\boldsymbol{\Sigma}$ is the cluster covariance.
%\[
  %u_{ig} = \frac{v+p}{v+(\boldsymbol{x_i}-\boldsymbol{\mu_g})^{\top}\boldsymbol{\Sigma}_{g}^{-1}(\boldsymbol{x_i}-\boldsymbol{\mu_g})}
%\]

It is worth noting that one issue with the Mahalanobis distance metric in detecting outliers, is that outliers influence and may greatly inflate the covariance matrix which is itself used in calculating the Mahalanobis distance.
To address this circular dependency, packages such as the \Rpackage{robustbase}, use \gls{LOO} methods to identify outliers which have high leverage on the covariance estimation.
Another outlier metric, commonly used in linear regression is Cook's distance, which returns the leverage of every data point $i$  on the estimation of $\hat Y$ when the point is left out:
\[
D_i = \frac{ \sum_{j=1}^n (\hat Y_j\ - \hat Y_{j(i)})^2 }{p \ \mathrm{MSE}}
\]
where $p$ is the number of parameters in the model, $n$ is the number of datapoints, MSE is overall mean square error in the model, $ \hat Y_j $ is the predicted value for datapoint $j$ and $ \hat Y_{j(i)} $ is the predicted value when datapoint $i$ is excluded.
%left out of the model.

In higher dimensions, outliers are harder to identify as data tends to be sparse, but also because they can be outliers without being outliers in any single dimension.
Reducing the dimensionality, using for instance \gls{PCA} or \gls{viSNE}, and increasing the sample size can facilitate outlier detection.


\subsection{Between sample noise}


%A big challenge to the analysis of biological data, which greatly complicates identifying, matching and comparing clusters across multiple samples, is between sample noise.
Reproducibility is a big challenge in flow cytometry and biology in general.
As seen in \Cref{chapter:il2ra}, within-individual variation, as ascertained from biological replicates, greatly compromises statistical power in detecting between-individual effects.
In flow cytometry, differences in cell treatment can lead to very different scatter patterns, as illustrated when comparing the scatter profile of the sample in \Cref{figure:il2ra-manual-gating-strategy} in \Cref{chapter:il2ra} to that of \Cref{figure:tony-cd4-gating} in \Cref{chapter:il2}.
%Staining in flow cytometry is notoriously noisy, so that
However, even when experimental conditions are kept constant,
%for example in flow cytometry using the same fluorochrome-antibody panel and the same \gls{PMT} voltage,
the shape and location of clusters across experiments can change.
%However, for a given panel and experimental protocol, one would not expect the clusters location to be comparable  to move much on the scatter channels since the morphological attributes of the cells are unlikely to be dependent on staining titration.
%Noise can be due to batch effects, for example staining discrepancies in flow cytometry, insufficient number of parameters to distinguish clusters, or because of sampling variation in rarer clusters.
We attempt to correct for these batch effects using normalisation.
%Normalisation can facilitate this task by accounting for batch effects before clustering.

Normalisation involves first matching certain features of the data across samples, and then transforming the data such that samples are comparable in a biologically meaningful way.
When identifying a fixed number of features across samples, such as modes in the density function, the first part of normalisation can rely on clustering algorithms such as K-means, K-medoids, \gls{GMM} or sliding window approaches.
%If k is unknown, the highest peaks can be selected across samples in the hope that these are the same across samples.
%Here normalisation is comparable to mode or bump hunting on the density function.
K-medoids was used successfully in \Cref{chapter:il2ra} when gating bead data from different days, and again in \Cref{chapter:kir} to qPCR data for pooling data across different plates.

In \Cref{chapter:il2ra}, the linear transform applied to align the peaks of the bead data was later applied to the identified clusters in the biological data, in order to make \acrfullpl{MFI} comparable across batches processed months apart.
%In \Cref{chapter:il2ra}, I showed that it possible to account for this time effect using beads.
%However as seen in \Cref{chapter:il2}, beads did not capture variation on a shorter time scale like seen in surface staining and in particular intra-cellular staining.

%Typically, normalisation is applied to univariate data, while clustering is applied to multivariate data.
%Meta-clustering which involves matching clusters across samples can also be considered to be a form of normalisation/clustering.
In \Cref{chapter:kir}, I applied between sample normalisation to qPCR datasets to enable pooling, followed by joint calling.
I first identified the copy number peaks in the $\Delta$Ct of qPCR plates of the two most common groups.
The peaks were then aligned with a linear transform across plates.
Pooling permitted the identification of the 3-0 group in \Cref{figure:Figure-1}.
In genotyping, normalisation is greatly facilitated by having matching probes across samples or spike-ins.
%On the other hand, in flow cytometry, there is much variation in the count of cells per sample.

In flow cytometry, there is a much larger number of cells than markers per sample, so clustering is typically done within each sample independently, followed by matching of clusters across samples.
%Normalisation can be done after the clustering to match and align the MFIs across samples.
%However, while this transform is appropriate for bead data which is not expected to change, it is obviously not appropriate when the MFI varies with genetic differences or ex-vivo stimulation.
%It can nonetheless be useful as a means to match populations across samples when the relative cell proportion is the parameter under study.
%Which is why in \Cref{chapter:il2ra}, beads are used to correct the sample MFI.
However, normalisation using peak alignment can also be done to align samples before the clustering, so that the same gating can be applied to all normalised samples.
This allows for pooling of samples, to aid the identification of rarer cell populations \citep{Hahne:2009hl}.

Stability of stains is a big challenge and biologists often have an intuition of which markers are stable, and objectively, it does appear that certain stains are much more stable than others, be it due to the antibody specificity, the chemical stability of the fluorochrome, the staining protocol or the thoroughness of the lab technician.
In \Cref{chapter:il2ra}, I was able to improve the repeatability of the CD25 MFI by correcting long-term fluctuations thanks to bead normalisation.
In \Cref{chapter:il2}, the variation in pSTAT5 MFI was not adequately captured by beads, possibly due to titration issues, so I had to resort to various other normalisation approaches, none of which performed particularly well.



%and become harder to call when the number of dimensions increases (they are not linearly separable in all dimensions).
In the context of flow cytometry data, the manual gate hierarchy contributes prior information about the expected relative frequency of the different types of cells and their relative marker expression, but their absolute marker expression is generally not readily comparable across samples and requires normalisation as shown in \Cref{chapter:il2ra}.
%This is why thresholds are often used to dichotomise the data into negative and positive subsets, however the problem remains of where to adjust the threshold per sample.

From my experience, normalisation of raw flow cytometry data using peak identification is as hard a problem as clustering due to the multimodality and the level of noise in these data.
%In \Cref{chapter:il2}, I found that the peaks could not be identified reliably using a sliding window approach because of spurious peaks caused by debris.
%Choosing the right window-size parameter for peak finding algorithms is channel specific and not trivial.
Also this type of univariate clustering of each flow marker independently does not exploit the correlation which exists between markers.
Finally, mismatching of peaks in the alignment is more detrimental to repeatability than no normalisation, as the wrong clusters will be aligned.

While in genetic data, the number of probes across arrays is constant and identifiable, flow cytometry data can contain very different number of events between samples and the distinction between cell populations is often blurry.
Normalisation methods applied to flow cytometry data must thus account for the sampling variation as well as staining discrepancies.
Distinguishing between staining noise and actual differences in cell biology requires a certain level of prior knowledge which is context dependent and difficult to implement algorithmically.
%Subsequently, a sample from the same individual taken and analysed on a different day, can have a very different profile.

\subsection{ Small number of samples }

%Typically, immunostaining flow cytometry experiments contain much smaller number of samples than genetic case-control experiments since, on one hand, samples are a limited resource and on the other, sample preparation and running tubes on the flow cytometer are an expensive and time-consuming process, while genotyping has become highly automated with an associated decrease in costs over the last decade.
Typically, immunostaining flow cytometry experiments contain much smaller numbers of samples than case-control genetic experiments since blood and \gls{PBMC} samples are a limited resource and sample preparation and running tubes on the flow cytometer are an expensive and time-consuming process.
Also, experiments undertaken in flow cytometry are often pilot experiments or tubes run to test and optimise panels, and therefore often not complete datasets.
These pilot experiments may be implemented with varying degrees of thoroughness and hence are generally poorly comparable.
Consequently, there are few flow datasets sufficiently well-powered to detect the more subtle effect sizes expected in complex diseases.

When dealing with the relatively small numbers of samples available in flow cytometry, I believe data quality is more influential than the methods used to process the data: sophisticated methods are no replacement for good data.
In order to be able to make the judgment call between "good" and "bad" data, understanding of the experimental context and the underlying biology is required; for example which cell populations to expect and their relative position and frequency.
In genotyping, calling a genotype based on only a few samples is sometimes possible, since the absolute position of the signal clouds can be estimated due to the stability of the DNA molecule, the standardisation of DNA preparation protocols and of SNP arrays \citep{Di:2005uj,Giannoulatou:2008ty}.
%Furthermore the population frequencies of many variants have been estimated in previous studies, providing some prior of what proportions of wild-type, heterozygous and homozygous, to expect in a new sample of unrelated individuals drawn from the same population.
On the other hand, in flow cytometry, the average protein expression of cell populations, as measured by the \gls{MFI}, is generally not directly comparable across experiments, so it is difficult to predict where a cell population will fall.
Instead, as is done when sorting cells, a few events need to be collected first in order to estimate where to draw gates.
Also, the frequency of cell populations can differ widely between individuals, and the ratio of cell populations, for example naive to memory, may change in the course of an individual's lifetime with exposure to environment. 
However, clusters can be defined in relation to one another, for example memory cells are lower for \protein{CD45RA} and higher for \protein{CD25} than naive cells (\Cref{chapter:il2ra}).

This prior knowledge is acquired through the experience of having seen a large number of samples and is difficult to encode in an automated method.
Processing in larger batches, or perhaps reducing the human involvement in flow cytometry, could be a first step towards automation.
%http://www.aber.ac.uk/en/cs/research/cb/projects/robotscientist/
This will become a necessity as the number of samples grows.

%Futhermore the fluorochrome antibody mix used per tube costs in the order of \textsterling20
%the method of operating flow cytometer is still quite manual in most labs.
In fact in larger flow facilities, such as the Vaccine Research Center at the National Institute of Allergy and Infections Diseases, some of the sample processing has been partially automated with robotics and can consequently process hundreds of samples a day.
Automatic methods are more pervasive in those labs since manual analysis is no longer a viable option.
%Within our lab, samples from longitudinal experiments are more common and come in over a long period of time and need to be analysed on the day or frozen.
%Furthermore, normalisation beads or controls are not consistently used which complicates analysis.


\section{ Moving from manual gating to automatic gating }

Despite the clear advantages that automatic gating promises over manual gating, fully automated gating may not always perform as well as expected due to the level of noise and the small number of samples.
%Various cell preparation or staining artefacts in flow cytometry can make cell populations indistinguishable, and beyond a certain threshold of uncertainty, a sample yields little information.
%Spillover introduces artificial marker correlation and can increase or decrease the fluorescence intensity of cell populations.  
Realistically, the move to automation is likely to be incremental, for example by replacing the sequence of univariate or bivariate gating steps in the process, by automatic methods.
As was seen in \Cref{chapter:il2ra}, the one-dimensional sequential top-down gating strategy can easily be implemented as an algorithm using mixture models or bead-derived thresholds.  
Perhaps the immediate goal of automatic gating shouldn't be to supersede manual gating but rather to complement it.
Until the number of samples is sufficiently large, these methods can benefit from the prior knowledge that manual analysis contributes, while providing more objective analysis.

My analysis of flow cytometry data has brought to light many issues surrounding the division of skill and the difference in thinking between data generation and data analysis.
Here I will present steps to be taken towards adopting these more targeted approaches routinely.

\subsection{Agreeing on standards}

While people generally agree that standards are key to improving reproducibility, there is often disagreement about which to adhere to.
However, inconsistencies as trivial as naming conventions may waste precious man-hours for the person analysing the data, if they were not involved in the data generation.
This is partly because the \acrfull{FCS} does not contain sufficient metadata to understand the context of the experiment.
Instead the name of the FCS file is typically used to map the sample back to the donor in order to retrieve covariates such as disease status, age, sex or genotype.
However, when the naming and documentation is incomplete, this makes it very hard to automate the analysis of these data.
Some guidelines have been set out by \citet{Lee:2008ed} in an attempt to define the minimum information to be provided for a flow cytometry experiment (MIFlowCyt):
\begin{itemise}
\item The experiment overview, which includes the purpose of the experiment, the experiment variables, conclusions and the quality control.
\item Information about the sample such as the source, the material used, the treatment of the cells and the reagents.
\item Instrument configuration details.
\end{itemise}
%Instrument identification Fluidics configuration
%Optical configuration
%Electronic configuration
%Data analysis
%List-mode data
%Compensation
%Gating
%Descriptive statistics
Human errors such as typographical errors or inconsistencies in the naming of FCS files, fluorochromes and antibodies across experiments, are, in my experience, very time-consuming and distracting from the analysis.
Even when the FCS file does contain metadata such as the date of the experiment, it does not always match what is given in the name of the file.
%Some minor issues for example, when experiments are done over night the dates might mismatch
% Niclas Thomas Very true, maybe worth making more of this point... - naming conventions for fluorochromes and antibodies sometimes vary from one experiment to the other, even when the same scientist has performed the experiment (eg naming CTLA4 in one experiment then CTLA.4 in another makes it necessary to either code in extra checks, or in opencyto the antibodies are named in a csv file, making it more difficult to overcome this lack of consistency).

I believe part of the solution is to involve the person generating the data in the analysis, so that they can appreciate the implications of naming inconsistencies.
Another part of the solution is to encourage automation of these more tedious tasks, as is being done in certain labs which use robots to feed the flow cytometer and barcoding to name the samples.
%At least automation is consistently and predictably wrong.
At the same time, the data analyst also needs to have an appreciation of the quality of the data and the purpose of the experiment.
For instance, in flow cytometry, it is important to remember that a large number of events are debris of no biological interest and can be excluded based on side and forward scatter.
Likewise artefacts can arise from staining, sample preparation (permeabilisation) or correlation of flow markers due to spillover.

While these observations are not specific to flow cytometry analysis, I would argue that these issues are more striking because of the flexibility flow cytometry offers both in terms of generation and analysis, and how little is known of the underlying cell populations.
In flow cytometry, the standardisation of reagents and operating procedures between laboratories is of crucial importance \citep{Maecker:2012gl}.

\subsection{Extending manual analysis}

In order to encourage the adoption of more rigorous computational gating methods, these tools need to be made more accessible to non-programmers.
One way to achieve this is by building on the FlowJo manual analysis.

%I think completely removing the two dimensional visualisation for example might be too alienating given how long this type of visualisation has been used.
%which remains the main tool used by immunologists for identifying groups of cells in flow cytometry.
%There is also a growing need for non-proprietary software which integrates well with the manual analysis.

A first step could be to use the manual gates as a template but automatically adjust them per sample.
FlowJo already provides a basic version of this feature, known as "magnetic gate", that moves gates to accommodate the maximum number of events.
X-Cyt \citep{Hu:2013bg} takes this approach further by using the mean and the covariance of the manually gated populations as initial starting parameters to a clustering algorithm, and then applying an \gls{EM} algorithm to refine the parameters to better fit the data.
%assuming an elliptical gate
I also tried a similar approach in \Cref{chapter:il2} since manual gates were not available for all samples.
First, I let the mean of the ellipse be influenced by the data while the covariance was set as fixed.
I then obtained a classification by defining a threshold on the Mahalanobis distance (\Cref{equation:Mahalanobis}) above which points were excluded.
This worked reasonably well unless there was too much overlap with another cluster, in which case the gate was pulled towards the wrong cluster.

Implementing this approach relies on extracting gate coordinates from FlowJo workspaces files.
%The unit of work in FlowJo is the workspace configuration from which FCS files are first loaded and then gated.
%The workspace also saves the cell populations statistics which need to be updated when the gates move.
Unfortunately, parsing FlowJo workspaces was not straightforward, since the format is poorly documented and not stable between releases.
Although there are several BioConductor packages designed to import and parse FlowJo workspaces, \texttt{flowUtils}, \texttt{gatingML}, \texttt{flowJo}, \texttt{flowWorkspace}, I have generally found that the R/FlowJo interface was not very reliable: for example \texttt{flowWorkspace} parsed the workspace without errors but the calculated statistics might not match the ones returned by FlowJo.
%This is probably why \contributor{Vincent Plagnol}
An alternative would be to develop a bespoke XML parser to extract gates from FlowJo workspace files, but this approach is laborious as it requires in-depth knowledge of the FlowJo XML schema, which changes on each new release.
%Additionally, I found the gate coordinates in FlowJo to be imprecise when plotted in R because FlowJo applies its own binning on the data.
%In fact, I found a serious bug in FlowJo: just loading an FCS file into FlowJo and reexporting it changes the data as some sort of binning is applied.

In the end, a simpler solution may be to use the CLR files which are simply the classification results from FlowJo.
Unfortunately, the current FlowJo implementation exports these files as text, which results in very large files, making the exporting impractical. 
In fact, FlowJo crashed or hanged on numerous occasions when trying to accomplish this simple task.
In the end, I had to resort to exporting only a few CLR files, from which I could estimate the gate coordinates.
One method of approximating gate coordinates is to calculate the mean and covariance of a CLR cluster and to use the Mahalanobis distance, hence approximating the cluster with an ellipse.
It is worth noting that a column naming standard is needed for the CLR format so that the position of the gate in the hierarchy, as well as its dimensions, can be retrieved.

Another interesting application of manual gates is to use them to define priors.
Specifically, when several samples have been manually gated, the mean and covariance of the gate across samples can be used to define the mean and covariance hyper parameters of the priors in the mixture model to guide the parameter estimation, as implemented in the \Rpackage{flowClust}.
I found this approach gave good agreement with manual gating for CD4\positive lymphocytes on forward, side scatter and CD4 on the dataset from \Cref{chapter:il2ra}.
%If several samples have been manually gated then these can be incorporated in the definition of the prior on the parameters of the mixture model.
%FlowClust as opposed to Mclust offers the option of defining per component priors.
%This threshold can be defined by using a chi squared distribution in the one dimensional case.

When FlowJo manual gating is not available, an alternative using R is to draw polygons on the R display using the \Rfunction{locator} and then using the \Rfunction{in.polygon} to extract points which fall within the polygon.
This is an approach that I used in \Cref{chapter:il2} to emulate \contributor{Tony Cutler}'s gating.
There also exist some commercial alternatives to FlowJo  such as \texttt{ADICyt}\footnote{\url{http://www.adinis.sk/en/products/bioinformatics-and-data-processing/adicyt.html}} and \texttt{Infinicyt}\footnote{\url{http://www.infinicyt.com/}}, but, at the time of writing, these are not nearly as widely used as FlowJo.
Even though commercial alternatives exist, I still believe there remains a gap in the market for a new open source and extensible piece of software which reconciles manual, supervised and unsupervised flow cytometry analysis, and provides further multidimensional visualisation techniques.
At the time of writing, a promising candidate is the \Rpackage{openCyto} which integrates various other R packages, such as \textsf{flowClust} and \textsf{flowDensity}, in order to offer access to a core library of automated gating approaches.
%to implement a unified approach to automated gating.
%can integrate various gating algorithms to define a gating pipeline.
%Niclas Thomas:
%However, it still necessitates some programming knowledge in order to be used successfully.
Although some programming knowledge is required for creating user-defined extensions, much of the programming requirements are removed by using spreadsheets to define the sequential gating process, known as gating templates.
%Gating templates are then parsed to R via one simple command, limiting the depth of programming knowledge required by the user.
Once gating has been completed, summary statistics and plots are easily obtained, and the raw fluorescence measurements can also be extracted for further analysis.
While the number of studies that have adopted \textsf{openCyto} is low at the time of writing, I envisage a large increase in its use in the near future.

%\subsection{Assessing performance}
%
%In order to ensure the quality control of these methods so that we exclude outlying samples.
%Benchmarking automated analysis against manual, we are not exploiting the true power of automated algorithms which is to teach us new biology or to put back into question our hierarchical view of immunology.
%As an exercise, I ran flowClust unsupervised with a large number of clusters and then picked the cluster which gave the best association with each SNP.
%This is an idea which was explored with flowMeans \citep{Aghaeepour:2010fv}.
%The issue however is the metaclustering step of matching clusters across samples is not trivial especially in the presence of batch effects.

%Therefore clustering is not an automatic task, but an iterative process of knowledge discovery or interactive multi-objective optimization that involves trial and error.  
%It will often be necessary to modify data preprocessing and model parameters until the result achieves the desired properties.  
%Clustering can also be viewed as a latent variable problem where the cluster labels are considered to be the missing data.



\section{Conclusion}

%Niclas Thomas:
%Becher:2014
%one main point on conclusion - you mention automated analysis will come into its own with larger samples and/or parameters, but don't go into much more detail - I think you really should mention that Cytof is redefining traditional flow cytometry, allowing unprecedented numbers of parameters (e.g. 38 http://www.nature.com/ni/journal/v15/n12/abs/ni.3006.html) and so the advent of automated gating may not be too far off

Larger datasets have allowed us to see finer biological variation, both in genetics and cell subsets, than previously possible.
Also, increasing the number of dimensions, by combining different kind of experiments, for example qPCR coupled with genotyping (\Cref{chapter:kir}), have helped uncover patterns which might not have been visible even at larger sample sizes.
%or inreasing the number of measured parameters, such as markers in flow cytometry,
%allowed me to discover a SNP predictive of \gene{KIR3DS1}-\gene{KIR3DL1} copy number which would have not found even at larger sample sizes.
These large datasets lend themselves to more sophisticated analytical methods, as was illustrated in \Cref{chapter:il2}, where unsupervised clustering on pSTAT5 dose-response, helped uncover previously overlooked responsive cell subsets.
In theory, these large datasets can be analysed using sophisticated models with a large number of parameters to account for all the intricacies of the data, as they are less prone to overfitting, but often in practise, simpler methods yield similar performance and are much more efficient.
Such methods can also be combined to reach a consensus, and this popular machine learning approach, known as boosting, increases performance at the expense of interpretability.

Interpretability is one of the concerns biologists discouraging biologists from adopting these computational methods.
For example, while mixture modelling approaches are conceptually close to manual gating, probabilistic cell types may not be intuitive to biologists, so their true power is not fully exploited as hard cut-offs are applied early in the analysis process (see \Cref{chapter:il2ra}).
Biologists prefer manual gating as it gives them the flexibility to draw exclusive gates whose contours can be made arbitrarily complex.
This freedom, however, exarcebates the disagreement in standards and definitions in immunology, and reproducibility of results.
Discrepancies in gate positions are unlikely to make much of an impact on the MFI and relative proportion of common cell populations, but these make a big difference in rarer cell populations, such as regulatory T cells, where the effects are much smaller.

%This can occasionally be achieved by centering the ellipse on the mean of the datapoints which fall within the gate.
While, in my and other computational biologists' opinion, automated clustering ought to be applied more widely to flow cytometry data, the use of manual analysis is likely to persists where the number of samples and parameters permits it.
However, as the number of samples and parameters continue to grow, as new single-cell biotechnologies such as CyTOF \citep{Becher:2014} and Drop-seq \citep{Macosko:2015} become widespread, biologists will need to relinquish more control to the computer.
%Hence it is important to continue developing these methods.
%These methods will require some level of expertise and decent visualisation to guide and reassure the user.
%Over-reliance on visualisation can mislead the analysis of high-dimensional because clustering is always projected back to two dimensions or linear combinations of dimensions.

The automatic gating community is growing stronger, with a number of contributions to BioConductor, and the GenePattern web interface maintained by the Broad Institute \citep{Kvistborg:2015}.
In particular, two labs, Raphael Gottardo at the Fred Hutchinson Cancer Research Center in the USA and Ryan Brinkman at the Terry Fox Laboratory in Canada, have been central in developing automatic gating software and bring together the automatic gating flow cytometry community as part of the FlowCAP challenge every year \citep{Aghaeepour:2013dg}.
% I admit there has been a lack of simulations but flow data is hard to simulate.
% it's not just simply labelled data that can be permuted and anyway the datasets are too large for this approach to be practical
% simulate small responsive population?
%On sufficiently large datasets, all methods tend to be equivalent.
%In a Bayesian setting, as the datasets are growing larger, the likelihood computed on the data will have a much stronger influence than the prior.
%Methods should be simple but no simpler.
%Our mission should not be to foreseee the future but to enable it.  
%How much we chose to automate depends greatly on the experiment but also more pertinently on the volume of data generated.
%However poor staining or instrumental configuration can lead to unexpected distributions.
%In \Cref{chapter:il2}, the permeabilisation protocol made the staining noisy, so that the MFIs were not reproducible.
%But different computational tools could be used to highlight different populations.
One of the key questions that this community addresses is how we should benchmark these methods.
%Philosophically this is perhaps what still makes data analysis an art, is that there is still some mystery since we haven't discovered all the rules yet.
If our benchmark is comparison to manual gating then clearly no method can ever outperform it.
Independent benchmarks such as repeatability are fairer, but perhaps a more general benchmark could be the utility of the clustering outcome, for example, whether it is predictive of diagnosis, prognosis, or if it correlates strongly with genetic variation or some other covariate under study.
Of course, we will need to account for multiple testing, given the large numbers of cell populations and phenotypes which will be tested with this method, and their correlation, given the interdependency between cell phenotypes \citep{Roederer:2015eu}.

%In my opinion, two aspects of flow cytometry analysis which are worthy of further work are normalisation and selection of an optimal transform.
%Both of these could be included as part of the clustering step.

Finally, my view is, that in the context of flow cytometry, we will only reap the true fruits of automated analysis once the number of parameters and samples has grown considerably, although this may require earlier steps in the pipeline, such as sample processing, to be better standardised and automated, so that the signal may rise above the noise.

%We might even see Cytobank accumulate samples like ArrayExpress.
%Finally, to conclude on a more philosophical tone from on my favourite french authors:

%Despite huge technological advances flow cytometry is still a very noisy technology and while bespoke analysis methods can be developed, larger sample sizes may be necessary to overcome the multitude of unwanted batch effects ranging from antibody stickiness to sample preparation.


%\begin{center}
%\vspace*{\fill}
%{\fontfamily{frc}\selectfont \foreignlanguage{french} {
%L'avenir n'est jamais que du présent à mettre en ordre, tu n'as pas à le prévoir, mais à le rendre possible.
%%Pour ce qui est de l'avenir, il ne s'agit pas de le prévoir mais de le rendre possible.
%}}
%
%-
%{\fontfamily{frc}\selectfont \foreignlanguage{french} { Antoine de Saint-Exupéry }}
%\vspace*{\fill}
%\end{center}
%









