\chapter{Discussion}


\section{ Important practical considerations with computational clustering methods }

Throughout my thesis I have considered methods of computationally analysing flow cytometry and genetic datasets.
The key motivation was that these methods would be more efficient, consistent, objective and better at dealing with uncertainty, than manual analysis of the data.
In the case of flow data, automating gating can also provide the added benefit of formalising manual gating, which remains to this day a very subjective process.
The hope was that these methods would improve on the reproducibility of results, thus enabling more powerful association testing.
Assessing how well computational methods perform is a matter of debate and it is safe to say very much data-dependent.
However, I will attempt here to give an overview of the important practical and theoretical considerations of these methods, in the context of my thesis.

\paragraph{Memory usage and running time}
Two crucial practical properties are their memory usage and running time.
While the applications that have been studied in this thesis don’t require real-time analysis, there are still some practical limitations on the amount of memory and compute time demands.
It is a well known fact in computer science that memory can often be traded for running time.
For example, methods which rely on global knowledge of the data, such as computation of the complete pairwise distance matrix,
have a large initial memory footprint ( data matrix of size $\frac{N \times (N-1)}{2}$ ), but only require one pass of the data.
A solution to the clustering solution is reached within just a few computational steps, by for instance,
selecting a distance cut threshold on the dendrogram generated from hierarchical clustering to define K clusters. 
On the other hand, methods like K-means necessitate a much smaller memory footprint
since they only compute the distance of every point to the K cluster means
( data matrix of size $N \times K$), which is usually a much smaller number than the total number of points,
but on the other hand, several updates of the matrix are required until the cluster centers are fixed.
The choice of which method to apply is very much data dependent.
In ungated flow cytometry, data matrices  contain over a million rows, so the complete pairwise distance matrix is too large to fit in memory,
making these type of methods impractical.
I found that distance matrix computation was only feasible in subsets of the order of $10,000$,
like for example the CD4+ lymphocyte subset, or data downsampled using the SPADE or binning in \Cref{chapter:il2}.
The data downsampling as performed by SPADE, relies on first estimating the local density at each point and then preferentially thinning the data in regions of high density to even out the density across the whole sample.
The local density estimation step is computationally intensive as it needs to consider the distance from a single point to all other points in the sample to find the number of neighbours.
I found this step could be greatly sped with \gls{ANN} which uses the K dimensional trees (KD tree) lookup method.
KD-trees are a space-partitioning data structure for organising points in a k-dimensional space which makes for an efficient
way of storing a high-dimensional dataset to lookup proximal datapoints.
It can also be used for approximation to reduce the number of datapoints which need to be considered.
This approach can also be applied to mixture models \citep{McLachlan:2004uw}, but I haven’t had the opportunity of trying this. 

\paragraph{Consistency and accuracy}
Consistency can be defined as how often does a method return a similar result.
This obviously depends on how much the data has changed.
If a method is consistent, one would expect that small perturbations to the data should lead to small changes in the clustering outcome.
However, algorithms which rely on initialisation using random starting positions like K-means may reach different clustering solutions, even on the same data!
When running K-means I therefore had to pick the initial clusters or run it with multiple restarts.
Over all runs of K-means the solution which returns the lowest within-sum-of-squares is picked.
The accuracy of a method is defined based on how frequently the methods assigns the correct label.
Hence it relies on the existence of a test dataset, typically labelled using manual analysis or some other method.
In \Cref{chapter:il2ra}, accuracy was assessed by comparing the cluster proportions and means with those obtained from manual gating.
In \Cref{chapter:il2}, the accuracy was determined in terms of RSS of the pSTAT5 response.
In \Cref{chapter:kir}, I used qPCR labelled data to assess the prediction accuracy of the \gls{knn} classifier.
However, labelled data may not always be available especially in the case of flow cytometry data.
Also, in the context of flow cytometry, even when labelled data is available, this approach may not always be ideal,
as it is merely comparing the relative agreement between methods rather than the objective truth.
A sometimes more useful alternative is instead to assess the prediction accuracy with clinical outcome, case-control status or genotype.
This is usually implicitly measured when doing an association test.
The consistency-accuracy tradeoff is analogous to the variance-bias tradeoff in statistics.
Consistency and accuracy are typically measured on simulated data.

\paragraph{Interpretability}
While a method might be accurate and consistent, it may be difficult to interpret the results, much like a black box.
Making a model more flexible by adding parameters can obfuscate the relationship between the input parameters and the clustering output.
Random forests and neural networks are example of methods from which it is difficult to extract an interpretable model to justify the result.
This is an issue because
%generally people (and especially inquisitive minds like scientists) are not comfortable with applying methods they don’t understand.
as part of the iterative process of knowledge discovery, its important to understand what combination of features make objects distinguishable.
%Since fully automated methods are hard to assess.
Sometimes the additional information which is available from previous studies can be used to guide the clustering.


%\paragraph{Parallelisation}
%Provided the data can be split into independent chunks, then this feature can be exploited with parallelisation.
%By make astute use of parallelisation, one can marry local and global methods to make efficient use of the computational facilities
%available such as compute farms of many nodes with limited memory and processing power.
%In such a scenario, each compute node can work on subset of the data to compute intermediate results which can then be combined.
%In the context of flow cytometry however, samples were not split but analysed individually on each compute node.
%In the future, it may be possible to exploit the hierarchical nature of the gating to conduct parallel analysis of independent subsets.

%\section{Clustering with prior knowledge}
%\section{Challenges of computational methods}
\section{Dealing with noise by using prior knowledge}

Having gone through some of the practical aspects I had to consider when applying computational clustering methods,
I will now address some of the more theoretical challenges.
%One compromise is that by adding more markers to delineate cell populations we are increasing the number of potential clusters and thus the tail of distribution of cluster sizes
%Variation due to batch effects in the preparation or sample ascertainment,
%or insufficient number of flow cytometric parameters
Perhaps the biggest challenge in clustering of biological data which greatly complicates consistent identification of clusters across multiple samples is noise.
Noise can be due to batch effects, for example staining discrepancies in flow cytometry,
insufficient number of parameters to distinguish clusters,
or because of sampling variation for clusters containing few elements.

%and become harder to call when the number of dimensions increases (they are not linearly separable in all dimensions).
In cases when certain characteristics of the data are known, these can greatly assist in the clustering
especially with regards to identifying smaller groups.
For example, in \Cref{chapter:kir}, I used prior information about the copy number of the \gene{KIR3DL1}/\gene{KIR3DS1} genes, obtained from qPCR,
in order to identify clusters in the SNP dataset.
The \gls{knn} was used to classify unlabelled points (without qPCR data) from the vote of their K nearest labelled neighbours.
The optimal value of K was selected by minimsing the \gls{LOOCV}.
Another source of prior information in labelling the samples with qPCR,
would have been to use expected copy number group frequencies obtained from previous studies 
\citet{Jiang:2012cf} as prior group frequencies in the clustering of the copy number groups in the qPCR data.

In the flow data, the manual gates contribute prior information to guide the automatic gating.  
They yield information about the expected relative frequency of the different types of cells and their 
relative marker expression. Their absolute marker expression is generally not readily comparable across samples and
requires normalisation.
The established manual gating method involves obtaining gate coordinates from drawing gates in FlowJo and applying these same
gate coordinates across samples.
Manual gates can also provide initial starting parameters to a clustering algorithm,
by calculating the mean and the covariance of the gated populations.
%For example, the labelling in one sample to the mean and covariance of an ellipse which can then be applied to another sample.
Using an EM style algorithm, the parameters of the ellipse can then be refined to better fit the data.
This is the approach taken by X-Cyt \citep{Hu:2013bg} which uses manual gates as templates.
Also FlowJo has a magnetic gate feature which is described as "gate is as moved to accomodate the maximum number of events".

I have also tried this approach.
I have let the mean of the ellipse be influenced by the data while the covariance was set as fixed.
The Mahalanobis distance then allows us to go from a mean and covariance of an ellipse to a classification by defining a threshold
above which points are excluded from the ellipse.

If the manual gating is done in several samples then they can be combined to define the priors in the mixture model,
so to guide the parameter estimation, as is done in flowClust.
If several samples have been manually gated then these can be incorporated in the definition of the prior on the parameters of the mixture model.
%FlowClust as opposed to Mclust offers the option of defining per component priors.
%This threshold can be defined by using a chi squared distribution in the one dimensional case.

An outstanding problem which remains with supervised clustering however, is how to account for unlabelled data, or in the context of flow cytometry, nuisance clusters which are not part of the gating?
While in manual gating these points are discarded and do influence the clustering, automatic methods may consider all points in the dataset and consequently these points will influence the gating.
One solution is to define a background cluster but this is unlikely to work for multimodal distributions.
Another solution would be to allow for the creation of a large number of clusters to account for all the background clusters which are not part of the study.

Supervised clustering can be really beneficial in trying to identify rare subsets.
This is because rare subsets, by definition, may not always be visible in all samples.
%Futhermore another concern with clustering and statistics in general is distinguising rare subsets from outliers.


\paragraph{Advantages and disadvantages to a tree structure to gating}


The hierarchical view is attractive, and using a tree structure for gating presents a number of advantages.
From a pratical perspective it is efficient, since points are filtered at each step, reducing the dataset.
It agrees with the established nomenclature facilitating comparison to manual gating and interpretability.
Methods which use a binary structure are easy to interpret as they conform to the manual gating strategy which provides rules on how to obtain the leaf subsets.
Reduces ambiguity.

The cellular hierarchical view is so deeply engrained in the minds of certain biologists that presenting clustering results in a different order could perturb them.

On the other hand, some disadvantages of using the tree structure to gating are that it can easily miss relevant cell populations and imposes a directionality in cell lineages which may not always be correct.

In the same way that phylogenies change as new animal species are discovered, cell lineages should also be influenced by the discovery of new cell subsets.

A bias is also imposed on the ordering on the markers and thus their importance. 
It may be more ambiguous markers are gated on first.
The RF gives us some idea of the importance of the markers.

The construction of tress can be very sensitive to the underlying data and errors propagate.
Errors may be mitigated by introducing conditional probabilities at each step.
This becomes more of an issue as more subsets are defined. 
If no restriction are appied the rules can become over-complicated.


%\section{ Distinguishing between outliers and rare subsets }
\section{ Identifying outliers }

Outliers have an important influence on the mean of the data.
They lie far from cluster means, usually in a sparsely inhabited region of space.
Thus clustering methods and statistical tests which rely on mean or covariance estimation are sensitive to outliers.
Certain outliers may be clearly identifiable because their values lie at the extreme of the instrument range,
far from the rest of the data.
%Very large intensity values are usually considered as noise, but low intensity values are harded to call with certainty.
Others which lie within a more plausible range of values are harder to identify.
Also in higher dimensions, data points can be outliers without being outliers in any single dimension.
Generally in order to detect outliers, large sample sizes are required.
Reducing the dimensionality makes outliers easier to identify.
%There is a soft approach of dealing with outliers and a hard approach.

Flow cytometry datasets contain millions of events and many events are discarded as part of the gating process.
Some of them are real biological clusters which are ignored because they are not part of the cell populations under study:
for example the first gate drawn is usually on the lymphocytes whereas the monocytes and granulocytes are often ignored.
This this in part because they are no markers included in the experiment defined on these cell populations which would allow any further division.
Other events are discarded because they are outliers. These outliers occur commonly in flow cytometry datasets caused by debris and cells clumping together.
They can be found at the extremes of the value range on a given channel, but they may also be found with intermediate values and these are harder to spot and require large samples sizes to confidently call as outliers.
Model-based methods can account for these by defining a background mixture with a covariance defined on the entire sample which essentially mops up all points
which are not assigned with high posterior weight to any of the known components.
Mclust allows to define a noise component for that purpose.
On the other hand, flowClust  has an outliyingness parameter which is inversely proportional to the Mahalanobis distance from a point to a cluster.
%\[
  %u_{ig} = \frac{v+p}{v+(\boldsymbol{x_i}-\boldsymbol{\mu_g})^{\top}\boldsymbol{\Sigma}_{g}^{-1}(\boldsymbol{x_i}-\boldsymbol{\mu_g})}
%\]
Another method which does not rely on a background component is to exclude low density points from belonging to any of the components.
A metric for detecting outliers is the Mahalanobis distance, large distances imply a point is far from a cluster center:
\[
(\boldsymbol{x_i}-\boldsymbol{\mu})^{\top}\boldsymbol{\Sigma}^{-1}(\boldsymbol{x_i}-\boldsymbol{\mu})
\]
where $\boldsymbol{x_i}$ is the point, and $\boldsymbol{\mu}$ is the cluster mean and $\boldsymbol{\Sigma}$ is the cluster covariance.

One issue however is that the outliers influences and greatly inflate the covariance matrix which is used in calculating the Mahalanobis distance.
To address this, R packages like robustbase,
use leave-one-out methods to identify outliers which have high leverage on the covariance estimation.
Another outlier metric, commonly used in linear regression is Cook's distance,
which returns the leverage of every data point $i$  on the estimation of $\hat Y$ when the point is left out:
\[
D_i = \frac{ \sum_{j=1}^n (\hat Y_j\ - \hat Y_{j(i)})^2 }{p \ \mathrm{MSE}}
\]

In genetics, outliers may be harder to spot because of smaller sample sizes which also implies that we are also generally more reluctant to exclude samples.
In \Cref{chapter:kir}, a whole qPCR plate was excluded because the $\Delta$Ct distribution didn’t align well even after normalisation.
Later I found however that qPCR samples may possibly be scored independently of where they lie in relation to other samples by considering their background Ct median and spread.
This score could have been used to downweight these samples in the mixture model clustering which was used to assign them to copy number groups.

As well as techniques for visualising multiple dimensions.
Tools are being developed for visualisation of summary statistics in mutliple samples in order to spot outliers.
Boxplots are the typical visualisation tool for viewing univariate intensity data but in flow cytometry where we typically have multivariate data,
SPICE presents an overview using pie charts of multiple samples \citep{Roederer:2011hy}.
%An alternative visual representation are radarplots

They are however obvious outliers and less obvious outliers, which is why I believe a confidence score is appropriate in calling outliers.
While certain outliers can be confidently called independently because their values are at the extreme of the range of the instrument,
others are less clear and should not be discarded completely because in light of new data they may be assigned to existing clusters or
to new clusters.

In particular distinguishing between rare subsets and outliers is hard.
As illustrated in \Cref{chapter:kir}, the 3-0 copy number group is rare and these points could have been regarded as outliers had we not had strong prior evidence supporting the existence
of this group from previous studies.
However, having prior evidence supporting the existence of a cluster is not sufficient to identify the position of the cluster.
Priors usually determine expected proportions but give little information about the exact position and shape of the cluster, which are usually experiment specific.
This can be due to the stability of the biological sample but also the stability of the instrumentation and sample preparation.
In genotyping calling a genotype based on only a few samples is sometimes possible, since DNA is a very stable molecule and SNP arrays are highly standardised \citep{Di:2005uj,Giannoulatou:2008ty}.
In flow cytometry, the MFI of cell populations is generally not directly comparable across staining panels but instead clusters can be defined in relation to one another, for example memory cells are lower for \protein{CD45RA} than naive cells and so it is possible to label these populations as in \Cref{chapter:il2ra}).


\section{Prioritising normalisation or clustering}
%\section{To move data or to move gate: normalisation vs clustering}

%Sometimes it is better not to normalise and normalisation may hide the true biological effect.
%The question really boils down to whether we should normalise the samples or cluster each sample separately.
%Normalisation is effectively analogous to meta-clustering since we are attempting to match cell populations across samples.  
%Normalisation facilitates matching clusters across datasets but in doing can remove meaningful biological variation.  
%Theoretically if the data was perfectly aligned across days then gates should not be moved.  
%Threshold gates should not move unless they are relative to some internal control.
%The cell phenotypes which are usually measured are the mean intensity of a particular channel, or the percentage of cells.
%Depending on the shape of the distribution these can be more or less sensitive to the position of the gate.
%Normalisation removes unwanted experimental variation to make data comparable even when collected on different days,
%proccessed with different protocols or analysed with different instruments.
%However distinguishing between what is unwanted and what is meaningful variation relies ultimately on a bias-prone judgement call.
%As we have seen, different normalisation approaches make different assumptions about what is unwanted variation and the shape of the data.
%The actual choice of normalisation depends on the characteristics of the data we wish to compare.
%We have only considered the univariate density function here but identification can be extended to the multivariate case.
%Next let's consider normalisation of multivariate distributions.  
%The purpose of principal component correction is to remove unwanted correlation between samples.  
%When applying a linear transform to multivariate data the correlation between the dimensions is preserved since the multiplicative factors cancel out.
%However the covariance changes.
%Care must be taken when applying a non-linear transform.  
%The samples are related and we wish to exploit this structure without forcing data to be perfectly comparable.
%A less biased approach is to allow for different cluster location and shapes across datasets and instead incorporate the clustering in some sort of hierarchical framework.  
%Normalisation or meta-clustering is a necessary step when data is pooled across studies.  
%One issue with normalisation based on peak alignment is that the level of there is more noise variation than biological variation between samples so that in certain samples,
%so much so that in certain samples the peaks are only identifiable after preliminary gating of lymphocytes.  

Normalisation is a two-part process, the first part involves matching across samples,
the second part involves transforming the data so that the data is directly comparable across samples.

In certain situations,
%If matching across samples involves find similar features
the first part of normalisation is similar to clustering and can rely on related algorithms.
For example normalisation by peak-alignment can use clustering of univariate data to identify peaks (with known k).
This is the method I used in \Cref{chapter:il2ra} when gating bead data and the transform I defined to align the peaks of the bead data,
was later applied to the biological data to do bead-normalisation.
When K is unknown a sliding window approach can be used to estimate the number of modes peaks.
The highest peaks can then be selected across samples in the hope that these are the same across samples.
Here normalisation is comparable to mode or bump hunting on the density function.
The number of peaks returned is influenced by the span of the sliding window.
A large window span will tend to oversmooth the data, leading to fewer peaks while
a smaller window will call more peaks, but increases the chances of picking up spurious peaks.
Hence the number of clusters is controlled by the window span parameter, although the exact relationship between the parameter and the number of clusters is data dependent.

While normalisation is typically applied to univariate data whereas clustering is applied to multivariate clustering.
Meta-clustering which involves matching clusters across samples can also be considered to be a form of normalisation/clustering.

The question really boils down to whether, when we have sufficient data, should we normalise between samples or cluster each sample separately?

In genotyping, between sample normalisation is usually applied to enable data pooling,
since we have a large number of markers and too few replicate probes to do within sample clustering,
Clustering is then applied on the pooled data to call genotypes.
%Some within sample normalisation is also possible.
In genotyping, normalisation is faciliated by having the same number of markers per sample.
Features are directly comparable.
On the other hand, in flow cytometry, there is much variation in the count of cells per sample.

In flow cytometry, we have much larger number of cell than markers per sample,
so clustering tends to happen within a single sample in order to identify cell populations,
which are then meta-clustered across samples.
Normalisation can be done after the clustering to match and align the MFIs across samples.
However, while this transform is appropriate for bead data which is not expected to change,
it is not obviously not appropriate if the MFI is expected to change with genetic differences or ex-vivo stimulation.
It can nonetheless be useful as a means to match populations across samples when the relative cell proportion is the parameter
under study.
%Which is why in \Cref{chapter:il2ra}, beads are used to correct the sample MFI.
Normalisation can also be done before the clustering.
On one hand in can facilitate the clustering so that the gate positions need not be moved betwen samples as is suggested
in \citet{Hahne:2009hl}.
Or on the other it can allow for pooling of samples, which can be useful in order to identify rarer cell populations,
to improve the signal-to-noise-ratio or to define a transform which can be used to align the data.
However from my experience I have found that normalisation of flow cytometry is actually a harder problem than clustering
because of the level of noise inherent in these data.  Univariate clustering in flow does not make use of the full information
available. 
In my thesis, I have used normalisation both before and after clustering.
In \Cref{chapter:kir}, I have have applied normalisation to align copy number peaks in the $\Delta$Ct of qPCR plates before the clustering.
On flow data in \Cref{chapter:il2ra}, I was able to improve the repeatability of CD25 MFI by correcting long-term
fluctuations thanks to bead normalisation.
In \Cref{chapter:il2}, unfortunately the variation in pSTAT5 MFI was not adequately captured by beads,
so instead I attempted various other normalisation approaches.
Since pSTAT
First I attempted to correct for background 


\paragraph{Normalising after clustering: good for matching cluster across samples}

In the case when the centre of mean of the gate has moved, after a few E iterations of the EM algorithm for example, it possible to realign the gates on the data.
The advantage of this approach is that it doesn’t require all the data to be moved.
But normalisation is still required in the meta-clustering step to compare properties of the clusters.
Normalisation is effectively analogous to meta-clustering since we are attempting to match cell populations across samples.  
Normalisation facilitates matching clusters across datasets but in doing so can remove meaningful biological variation.  

The samples are related and we wish to exploit this structure without forcing data to be perfectly comparable.
A less biased approach is to allow for different cluster location and shapes across datasets and instead incorporate the clustering in some sort of hierarchical framework.  
Normalisation or meta-clustering is a necessary step when data is pooled across studies.


\paragraph{Normalising before clustering: good for pooling to find identify rare populations}

If applied before the clustering it enables data pooling to identify clusters across all samples.
Pooling may improve the coefficient of variation of populations but also facilitates
matching clusters across samples, a step known as meta-clustering.
Pooling is crucial for identifying rare groups which are difficult to identify within one sample.
If the alignment isn’t perfect, the clustering results can be refined across all samples.

Theoretically, if flow cytometry data was perfectly aligned across days then gates should not be moved, unless they are dependent on some internal control.  
For example a threshold gate on a positive subset may be defined in relation to
another population of cells.

As discussed in \Cref{chapter:il2ra},
the cell phenotypes usually measure, the mean intensity on a particular marker, or the percentage of cells can be more or less sensitive to the position of the gate,
depending on the shape of the distribution.


\paragraph{Conclusion}

Whilst normalisation is generally a necessary process for pooling data or comparing across samples, over correction can be detrimental to the analysis.
Normalisation is meant to be a simpler preliminary step to simply the clustering,
however, in flow cytometry, because of batch effects and unequal number of events per sample, normalisation can be as challenging as performing the gating.
%generally relies makes big assumptions and leads to major changes in the data.

However, when they are insufficient data points, pooling is necessary.
Clustering sometimes relies on pooled samples when they are not sufficient data points
to form clusters such as in the case of rare subsets like 3-0 in KIR (\Cref{chapter:kir}) or Tregs in flow cytometry data.
In this situation some form of normalisation is usually necessary.
Unfortunately, multivariate normalisation is not always trivial and relies on defining linear transforms. 



\paragraph{Some issues with normalisation}

Normalisation removes unwanted experimental variation to make data comparable even when collected on different days,
processed with different protocols or analysed with different instruments.
However distinguishing between what is unwanted and what is meaningful variation relies ultimately on a bias-prone judgement call.

As we have seen, different normalisation approaches make different assumptions about what is unwanted variation and the shape of the data.
The actual choice of normalisation depends on the characteristics of the data we wish to compare.

The Bayesian question of relative weight of prior vs data is very relevant to gating.
The position of the gates are the priors. Is it better to have an absolute definition/threshold or should the gates be relative to the data?
 
One issue with normalisation based on peak alignment is that the noise variation is greater than the biological variation between samples,
so much so that in certain samples, the peaks are only identifiable after preliminary gating of lymphocytes.

We have only considered the univariate density function here but identification can be extended to the multivariate case.

\paragraph{Normalisation of multivariate distributions}

%The purpose of principal component correction is to remove unwanted correlation which may exist between samples or GC content.  
When applying a linear transform to multivariate data the correlation between the dimensions is preserved since the multiplicative factors cancel out.
However the covariance changes.
Applying a non-linear transform also changes the correlation.
If the data is binned using flowBin or clustered using SPADE then a transform could be a mapping like Earth Moving to make both distributions of event proportions identifical across samples.

\paragraph{Combining transforms with clustering}

Choosing an optimal transform in flow data is not trivial.
Transforms tend to be channel specific and sometimes even sometimes sample specific.
%Per sample transform required like flowClust but using the sliding window approach.
%Some prior knowledge is required to know what makes a sensible density plot.
%Care needs to be taken not to introduce spurious peaks.
The FlowClust solution proposes estimating the Box-Cox parameter as part of the clustering using a numeric optimiser.
However the parameters need to be transformed back, for allowing for a different transform per sample.
This is an issue if comparing intensity data but obviously if comparing cell ratios then the transform is of little consequence.
Nonetheless the transform can have an important effect on the clustering result.

Clustering is an iterative method of discovery not a fully automated process.
%Some tuning is required
It will often be necessary to modify data preprocessing and model parameters until the result achieves the desired properties.  
Clustering can also be viewed as a latent variable problem where the cluster labels are considered to be the missing data.



%\section{ Method of determining phenotype affects power to detect genetic association }
\section{ Influence of normalisation and clustering on statistical association test }

Ultimately the goal of normalisation and clustering is to allow for association testing of genetics with biological traits.
%Decomposing the variance into tech
In complex biological traits, a significant proportion of the variation is explained by the cumulation of a large number
of small genetic effects.
%a multitude of smalls effects which cumulatively explain
%Most effects in biological variation are small, they therefore require large sample sizes and

The most studied factor which influences the statistical power to detect such effects is sample size, and we are witnessing ever-increasing sample sizes thanks to dropping genotyping and sequencing costs.
However, a less explored and harder to define factor is the accuracy with which traits are measured.
For example in \gls{T1D}, depending on the stage at which the disease is diagnosed, the phenotype can be quite different.
If the disease is diagnosed earlier then the associated phenotype may be recurrent auto-antibody positivity or if diagnosed late, high blood glucose.
In most diseases, case heterogeneity is a confounder and one which needs addressing by collection of additional clinical covariates during the recruitment of cases.
Since these disease status phenotypes are hard to define accurately and consistently,
%as I have touched upon in \Cref{chapter:intro},
this has motivated looking at better defined intermediate phenotypes (also known as endophenotypes), such as the CD25 cell phenotypes of \Cref{chapter:il2ra}, or the pSTAT5 exvivo response cell phenotype of \Cref{chapter:il2}, which might show stronger correlation with genotype as they are closer to the gene expression.
Cell phenotypes such as relative cell frequency over parent cell population or expression of surface markers can in theory be measured accurately with flow cytometry, but there are still factors which influence the accuracy at which these can be measured.
In particular in \Cref{chapter:il2ra} of this thesis, I have concentrated on how clustering can influence the relative frequency
and MFIs of the cell populations under study.
Other known factors are, staining batch effects as seen in \Cref{chapter:il2},
or long-term instrument sensitivity as seen in \Cref{chapter:il2ra},
which influence the cell population \glspl{MFI},
or sampling variation on each bleed which influences the cell frequency.

Using recalled individuals, I have assessed the repeatability of the cell phenotypes.

%Certain traits like height for example are not expected to change much after a person has reached a certain age
%is improved with large sample sizes but also if the ratio within-group to between-group variance.
If the repeatability is good then the within-individual variance is small which should increase our power to detect
small differences.

Consistency is an important factor in identifying clusters.
Decomposing the variance into within sample and between sample variation.

Subset analysis is also very important because of the phenomenon known as Simpson’s paradox whereby a trend that is visible in the group as a whole will often not hold in all subsets of the group and may even be reversed in certain subsets of that group.

When a hierarchical gating approach is taken, errors at the top of the hierarchy propagate down the tree.
%This approach is chosen for manual gating as it agrees with our very linear way of thinking.
This complicates an automated approach to stepwise gating since errors which occur in earlier steps of the gating may not be recoverable.

%In the same way wrongly gating samples can lead to outliers which can create false positive association due to their high leverage, fixed gating can lead to all samples having the same profile.

%\subsection{Reproducibility}

Reproducibility in flow cytometry is challenging.
While in genetic data, the number of probes across arrays is constant and identifiable,
flow cytometry data can contain very different number of events between samples.
Also distinguishing staining noise from actual differences in cells biology requires a certain level of prior knowledge which is difficult to implement programmatically.
Subsequently, a sample from the same individual analysed on different days can have a very different profile.
Biologists tend to have strong opinions on which markers are stable and objectively it does appear that certain stains are much more stable than others.
As seen in \Cref{chapter:il2ra}, this within-individual variation greatly compromises statistical power in detecting between-individual effects.

In long-running experiments when samples are analysed over a period of months, some noise may be attributed to variation in instrument sensitivity.
In \Cref{chapter:il2ra}, we showed we can account for this using beads.
However beads do not capture variation on shorter time scale or due to batch effects.
Staining and in particular intra-cellular staining, of internal markers such as STAT5 and FOXP3, is subject to batch and day variation.

Another approach is to align the peaks of the univariate distribution in the whole sample as was done for qPCR $\Delta$Ct values in \Cref{chapter:kir}.
However, in the flow cytometry samples we studied, we found that the peaks cannot always be identified reliably and choosing the right window-size parameter for peak finding algorithms is channel specific and not trivial.
Also mismatching of peaks in alignment is more detrimental to repeatability than not any doing normalisation.

%Therefore clustering is not an automatic task, but an iterative process of knowledge discovery or interactive multi-objective optimization that involves trial and error.  
%It will often be necessary to modify data preprocessing and model parameters until the result achieves the desired properties.  
%Clustering can also be viewed as a latent variable problem where the cluster labels are considered to be the missing data.


%Lessons learned
%Things I have learned: philosophical discourse

%\section{ future application of methods }
\section{ The future }

The methods discussed in this thesis suggests that there are many scenarios in which flow cytometry analysis can be automated.
There are however a number of outstanding challenges, some technical, some more philosophical, in applying these methods.

\paragraph{Seeing is believing}

Statisticians, biologists, we all like to visualise our data.
We rely on visual inspection as a quality control check, learning about properties of the data like for example if the data
is skewed or symmetrically distributed, looking for patterns, gene lists appearing in pathways, confirming clustering results, spotting outliers.
While visualisation works well for up to three dimensional data, information is lost when higher dimensional datasets are decomposed into a series of two-dimensional projections.
Clusters which exist in higher dimensions do not necessarily map to clusters in two dimensions.
%Clusters are not identifiable in two dimensions, in fact for clusters which are thus ignoring a large number of points.
%In fact K points are only garanteed to be separable in at least K+1 dimensions.
%An upperbound on the number of clusters in the data is the product of the univariate modes,
%whereas a lower bound is the maximum number of modes in one dimension.
This has motivated research into how to visualise high-dimensional data in two-dimensions with minimal loss of information.
In \Cref{chapter:il2}, I presented one these approaches, \gls{SPADE} which relies on a network visualisation of a dataset using a minimum spanning tree.
There are others which use more probabilistic approaches such stochastic neighbour embedding.
While I agree that visualising high-dimensional data using network representations can be insightful,
beyond a certain number of nodes, network visualisations quickly become uninformative.
Certainly in the case of flow cytometry some clustering or data smoothing is required to reduce the number of points.  
%There are many more which take more probabilistic approaches like stochastic neighbour embedding
%My opinion is that this fascination for visualisation boils down to a distrust of computer vision
%visualisation can also be misleading and lead to oversimplifications

%An important part of our job is to unearth trends or make them visually obvious.


\paragraph{Experiment vs analysis: a communication problem}

Analysing flow cytometry data has brought to light the many issues with the division of skill sets between generation and analysis of data.
Perhaps the greatest obstacle to standardising analysis is our inherent dislike of rules and standards which we may find superfluous and tedious.
However, trivial things like naming conventions may waste precious man hours for the person analysing the data or even to you if you return to your data after a long while.
Part of the solution is to involve the biologists, the person generating the data, so that they can appreciate the implications, another part of the solution is to encourage automation of these more tedious tasks, as is being done in certain labs which use robots to feed the flow cytometer.
%At least automation is consistently and predictably wrong.

On the other hand, perhaps statisticians need to have an understanding of: the quality of the underlying technology and the purpose of the experiment.
For example, in flow cytometry, a large number of events are debris of no biological interest.
Similarly some patterns may be simply staining artefacts or from sample prepartion (permeabilisation).

I spent a great deal of time on seeing if clustering can be fully automated, i.e truly non-parametric.
One approach is to select large K and then merge clusters together with flowMerge.
My approach was generally to select a K that gave consistent clustering results across samples.

While I am sure these observations are not only specific to flow cytometry analysis, I would argue they are more striking because of the freedom flow cytometry offers both in terms of generation and analysis, and how little is known of the underlying cell populations.


\paragraph{ Incomplete experiments and small sample sizes}

Often, experiments undertaken in flow cytometry are pilot experiments or tubes ran to test and optimise panels.
Pilot experiments
%may be motivated by a moment of inspiration and leads to the generation of flow data is often an adhoc thought process,
are often implemented with varying degrees of thoroughness and hence are generally poorly comparable.
Furthermore, normalisation beads or controls are not consistently used which complicates analysis.
While the FCS files are saved and analysed by the person who generated the data, the naming and documentation is incomplete
making it hard to automate the analysis of these data.
Since the FCS file does not contain sufficient metadata to understand the context of the experiment, the name of the FCS
file is typically used to map the sample back to the donor in order to retrieve covariates such as disease status, age, sex or genotype.
A lot of my work unfortunately has involved dealing with typos and inconsistencies in these file names.
Also channel names as given in the FCS files are not always consistent across experiments.
When the FCS file do contain metadata it does not always match the naming of the file.
For example the data as strored inside the FCS file did not match the date given in the filename.
In particular, FCS files do not contain sample identification information, which makes matching back to genotypes cumbersome and error-prone.
%Some minor issues for example, when experiments are done over night the dates might mismatch

Typically flow cytometry experiments do not have large number of samples because in most labs, sample preparation and running tubes on the flow cytometer are manual operations.
%Futhermore the fluorochrome antibody mix used per tube costs in the order of \textsterling20
%the method of operating flow cytometer is still quite manual in most labs.
However there are labs where these processes have been automated with robotics
%to feed flow tubes
and consequently may run thousands of samples a day.
Automatic methods are more pervasive in those labs since manual analysis is no longer a viable option.
%Within our lab, samples from longitudinal experiments are more common and come in over a long period of time and need to be analysed on the day or frozen.
Presently most flow experiments contain too few samples to do well-powered association testing.


\paragraph{ Stability of markers }

Staining in flow cytometry is notoriously noisy.
Even when using the same fluorochrome-antibody panel and the same PMT voltage, the shape and location of clusters is not stable.
While the scatter channels can be very noisy due of debris, for a given panel and experimental protocol,
the location of the clusters should not move much on the scatter channels because the morphological attributes of the cells should not be dependent on staining titration.
However, the treatment of cells can also lead to very different scatter patterns (example Tony vs Marcin).


\paragraph{ Flowjo interface with R}

FlowJo is the main tool used by immunologists for identifying groups of cells in flow cytometry.
The unit of work in FlowJo is the workspace in which FCS files are first loaded and then gated.
The workspace also saves the cell populations statistics which need to be updated when the gates move.

Unfortunately, parsing FlowJo workspaces in order to extract manual gates is not straighforward.
Although there are several BioConductor packages designed to import and parse flowJo workspaces, flowUtils gatingML, flowJo, flowWorkspace, I have found the R/FlowJo interface to not be very reliable, although the flowWorkspace parsed it without errors, the returned statistics calculated in FlowJo were wrong.
This is probably why Vincent Plagnol developed his own XML parser to extract gates from flowWorkspace files but this approach is time-consuming as it requires in-depth knowledge of the FlowJo XML schema which can change on each new release of FlowJo.
Furthermore the gate coordinates in FlowJo are often imprecise.
In fact, I found a more serious bug in FlowJo: just loading an FCS file into FlowJo and reexporting it changes the data!
In the end, I found the best solution was to export CLR files which are simply the classification results from FlowJo.
Unfortunately, instead of exporting these files in a memory-efficient compressed binary format, FlowJo exports them as text which results in very large files and makes exporting of all clustering results impractical.  FlowJo crashed on numerous occasions when try to accomplish this task.
Hence I resorted to exporting only a few CLR files from which I could estimate the gate coordinates.
However, one ommission from the CLR format needed to retrieve the gate coordinates is the dimensions in which the gate is defined.
One method of approximating gate coordinate is to calculate the mean and covariance of a CLR cluster and to use the Mahalanobis distance, hence approximating the cluster with an ellipse.
However, it helps to know in which dimensions the gate was defined.

Another solution to including manual gates in R without relying on their coordinates in FlowJo is to draw polygons on the R display and use the \Rfunction{in.polygon} to extract points in the polygon.

All in all, there is definitely a gap in the market for a new piece of software which reconciles manual, supervised and unsupervised flow cytometry analysis, and provides further multidimensional visualisation techniques.


%\paragraph{Inconsistencies in file naming} 
\paragraph{Mass cytometry}

Time of flight cytometry (CyTOF) is a biotechnology which combines mass spectometry with cytometry.
The throughput is not as high as fluorescence flow cytometry, but it pushes the number of markers which can measured up to 34 and potentially higher.
However, it cannot be used for sorting as the cells are destroyed when measured.
Also, as it does not report side and forward scatter, live/dead markers are used instead to spot debris.
The anlaysis of the type of datasets generated by this technology benefits greatly from the multidimensional visualisation techniques such as SPADEl \citep{Simonds:2011jh} and ViSNE \citep{Amir:2013jp}.


\section{Summary}

Larger datasets have allowed us to see finer biological variation both in genotypes and cell subsets than previously possible.
However, sometimes taking a different view of the same dataset, by doing a different kind of experiment, for example qPCR, or adding parameters, for example additional markers in flow cytometry, can help uncover patterns which might not have been visible even at larger sample sizes.
For example in \Cref{chapter:kir}, qPCR allowed us to discover a SNP predictive of KIR3DL1/3DS1 copy number.
Furthermore, even without doing further experiments, analytical methods such as unsupervised clustering can reveal previously unknown features.
For example in \Cref{chapter:il2}, unsupervised clustering algorithms analysing pSTAT5 response at different doses of proleukin uncovered responsive subset of cells we previously ignored.

Although large datasets can support methods with larger number of parameters as they are less prone to overfitting,
it is easy to fall into the trap of applying over complex analytical methods to account for all the intricacies of the data,
when in practice, simpler methods perform nearly as well and are much faster and easy to implement.
Simple methods can also be combined to reach a consensus and this popular machine learning approach known as boosting increases performance but often at the expense of interpretability.

Interpretability is perhaps one of the more important issues in order to encourage biologists to use these computational methods.
While mixture modelling approaches are conceptually close to manual gating, probabilistic populations are not always intuitive to biologists,
so their true power cannot be fully exploited as we end up applying hard cut-offs (see \Cref{chapter:il2ra}).
Biologists enjoy manual gating becomes it gives them the freedom to draw arbitrary exclusive gates whose contour can be made arbitrarily complex.
This freedom however comes at the cost of exacerbating the disagreement in standards and definitions in immunology.
While discrepancies in gate positions are unlikely to make much of an impact on the MFI and relative proportion of common cell populations,
they can make a big difference on rarer cell populations such as regulatory cells where the effects are much smaller.

In order to encourage the use of more rigourous computational gating methods, biologist need to be lured in by incrementally habituating them to these tools.
For example, a first step could be to use the manual gates but to allow them to move with the data.
Completely removing the two dimensional visualisation for example might be too alienating to some.
There is also a growing need for non-proprietary software which integrates well with the manual analysis.
The openCyto BioConductor package currently being developed by Gottardo
The automated methods need to complement the manual methods for now, so that the change from manual to automated happens gradually.
On the other hand by continuously benchmarking automated analysis against manual,
we are not exploiting the true power of automated algorithms which is to teach us new biology or to put back into question our hierarchical view of immunology.
As an example I ran flowClust unsupervised with a large number of clusters and then picked the cluster which gave the best association with each SNP.
This is the idea which was explored with flowMeans.  The issue however is the metaclustering step of matching clusters across samples is not trivial especially if
there is a lot of noise between samples.

%This can occasionally be achieved by centering the ellipse on the mean of the datapoints which fall within the gate.

While, in my opinion, automated clustering ought to be applied more widely in flow cytometry data analysis, data analysis can still be done manually as the number of samples and parameters are still manageable.
However, as the number of samples and parameters continue to grow, biologists will need to resort to fully automated method, hence itis important that we continue developing these methods.
These methods will require some level of expertise and decent visualisation to guide and reassure the user.
Although, over-reliance on visualisation can mislead the analysis of high-dimensional because clustering is always projected back to two dimensions or linear combinations of dimensions.
The automatic gating of flow cytometry community is strong with a lot of contributions to BioConductor and the GenePattern web interface from the Broad Institute.
In particular two labs, Raphael Gottardo at the Fred Hutchinson Cancer Research Center in the USA and Ryan Brinkman at the Terry Fox Laboratory in Canada, have been central in developing auto gating software and bring together the automatic gating flow cytometry community as part of the FlowCAP challenge every year.

In Stanford, Gary Nolan lab and at the CRI in the UK, mass cytometry is adopted to analyse cell heterogeneity in cancer.
These high-dimensional datasets require visualisation and Dana Pe'er group at Columbia University has devised various tools to do so such as ViSNE \citep{Amir:2013jp}.

I have also identified aspects of flow cytometry analysis which need more work such as normalisation and selection of an optimal transform.
Both of these can be included as part of the clustering step.
Logarithmic transform greatly influence the identification of lower intensity populations which overlap into the negative range.
The wrong transform can introduce splits giving rise to spurious cell populations.
This is why applying a straightforward arcinsh transform like is done in spade is wrong.
In FlowJo, the transform is selected visually, given the knowledge of what cell populations to expect.
The only existing automated  methods of optimally selecting a transform that I am aware of are flowTrans and flowClust.
FlowTrans assumes an underlying Gaussian distribution and used maximum likelihood to estimate the optimal transform parameter.
But there is still the question of which transform function to apply.
FlowClust applies a Box-Cox transform for which the lambda parameter is estimated as part of the ML estimation.
But as I showed in \Cref{figure:logicle-transform-w}, it is not always clear what transform to apply.

% I admit there has been a lack of simulations but flow data is hard to simulate.
% it's not just simply labelled data that can be permuted and anyway the datasets are too large for this approach to be practical
% simulate small responsive population?

Data quality is as important as the methods use to process the data: sophisticated methods are no replacement for good data.
However, being able to make this judgment call between good and bad data necessitates understanding of the experimental context.
This prior knowledge is acquired through the experience of having seen a large number of samples and difficult to encode in an automated method.

Processing in larger batches or perhaps reducing the human element in flow cytometry could be a first step towards automation.
%http://www.aber.ac.uk/en/cs/research/cb/projects/robotscientist/
However as the number of samples grows so will the need for computational methods and the gap between the biological of computational way of thinking will shrink.

On sufficiently large datasets, all methods tend to be equivalent.
In a Bayesian setting, as the datasets are growing larger, the likelihood computed on the data will have a much stronger influence than the prior.

In conclusion, fully unsupervised methods cannot be expected to deal with the level of noise possible in flow cytometry experiments, beyond a certain threshold of uncertainty, a sample is of little value.
Poor staining can make populations difficult to distinguish or can even make populations disappear.
Spillover introduces artificial marker correlation and can increase or decrease the fluorescence intensity of cell populations.  
With an in depth understanding of the patterns of noise, it is possible to develop more targeted approaches such as Adiyct but these run the danger of including outlier samples.
Ultimately deciding when or not to include an outlier sample is down to a judgement call, often by the person who generated the data, so in that regard subjectivity can persist and that can introduce bias.
%For the sake of interpretability I would advocate simple methods on complex data

%Methods should be simple but no simpler.
%Our mission should not be to foreseee the future but to enable it.  

How much we chose to automate really depends on the application and really comes down to a question of survival.
However poor staining or instrumental configuration can lead to unexpected distributions.
In \Cref{chapter:il2}, the permeabilisation protocol made the staining noisy, so that the MFIs were not reproducible.
But different computational tools could be used to highlight different populations.

The move from manual labour to automated manufacturing during the industrial revolution was partly driven by demand and since demand is forever increasing it is clear that we are moving towards automation until we reach the singularity.
It was the ability to break down the complex process of assembly into a series of simple basic steps tasks 
As was seen in \Cref{chapter:il2ra}, the 1D sequential top down gating strategy can easily be coded up as an algorithm using mixture models or bead-derived thresholds.  
The main difference here however is in the industrial process we know what to expect a shoe but in this process we may expect anything.
Philosophically that what still makes data analysis an art, we haven't discovered all the rules yet.
If our benchmark is comparison to manual gating then we can never do better than manual gating.
Here we have introduced a benchmark of the outcome which is repeatability.
Perhaps a more general benchmark which can be used is: it useful?
Is it useful for diagnosis or does it correlate strongly with genetics etc.
We will only see the true fruits of automated analysis once the number of samples has grown considerably.
but this might require earlier steps in the process such as the data generation to be further automated first.









