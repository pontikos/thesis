\chapter{ \label{appendix:fcs-data-format}Flow Cytometry Standard (FCS) Data Formats: FCS 2 vs FCS 3}

The objective of the Flow Cytometry Standard is to define a unified file format for flow data
that allows files created by one type of acquisition hardware and software to be analyzed by any other type.
The data format determines the range and the precision of the data stored.

The first Flow Cytometry Standard format for data files was FCS 1.0 \citep{Murphy:1984ev}.
The standard was later updated in 1990 as FCS 2.0 \citep{Anon:1990ce} and again in 1997 as FCS 3.0 \citep{Anonymous:vr}.
FCS 2.0 and FCS 3.0 are the current two main competing standards.


FCS 2 is a logarithmically compressed format which does not allow negative intensities.  
Instead negative values reported by the instrument are arbitrarily assigned the minimum value.
This leads to what is described as the log artefact: a pile up of intensities on the axes for low intensity values.
FCS2 data are integers in the range $1$ to $10000$ (4 decades).
FCS 3 on the other hand is closer to the raw data, covers a greater range and allows for negative values.
FCS 3 leaves more flexibility to the choice of transform.
FCS 3 are floating point numbers in the range $-211$ to $262143$ (8 decades)
FCS 2 is more trivial to process as it requires practically no post processing except for a log transform.
FCS 3 requires more careful thought as it leaves to us the compensation and the choice of a suitable transformation.
%In my work so far I have been using FCS 2 to facilitate comparison to the manual analysis but in the future I intend to use FCS 3 instead.  
%FCS 2 has compensation pre-applied whereas in FCS 3 compensation matrix is stored as part of the data format and needs to be applied manually.
%Unfortunately due to limitations of the fluorescent dyes and the instrument, the fluorescent signal measured in on channel is often a mixture of signals.
%This phenomenon is known as spectral spillover.
%The deconvolution of this signal is a process known as compensation.  The matrix solution is known as the spillover matrix
%and is usually a square matrix if there as many dyes as there are detectors.


