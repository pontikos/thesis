\chapter{\label{appendix:flowsorting}{Flow markers}}


\paragraph{CD3}
CD3 marks all T cells

\paragraph{CD8}
Cytotoxic T cells. Killer cells.

\paragraph{CD4}
A protein found on a subset of T lymphocytes.
Helper cells.
Response orchestrator.

\paragraph{CD31}
Largely present on naive CD4 T cells, is lost on maturation of naive cell after leaving the thymus.
%Recent thymic emigrants.

\paragraph{CD45RA}
The protein isoform lost on activation of naive CD4\positive and CD8\positive T cells.
It can be used to distinguish CD45RA high naive cells from CD45RA low memory cells.

\paragraph{CD127}
The alpha chain of the IL-7 receptor.
The IL-7 receptor is expressed on various cell types, including naive and memory T cells,
and usually expressed at higher levels on T effector and regulatory T cells.

\paragraph{CD25}
Better known as the \protein{IL2RA}, the alpha chain of the heterotrimeric IL-2 receptor.
High affinity binding of IL-2 requires all three chains of the receptor.

\paragraph{CD122}
The beta chain of the IL-2 receptor.

\paragraph{CD132}
The gamma chain of the IL-2 receptor.

\paragraph{CD56}
NK cell marker.


\paragraph{CD19}
Found on the surface of B-cells.
It is expressed on follicular dendritic cells and B cells.
It is a lineage marker which is lost on maturation to plasma cells.



\paragraph{CD69}
A protein induced by the activation of T lymphocytes and Natural Killer cells.
It is involved in lymphocyte proliferation and functions as a signal-transmitting receptor in lymphocytes.

