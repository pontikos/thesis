
\subsection{Prediction of pSTAT5 response with Random Forests} 
Instead of trying to identify clusters, an alternative is to try to predict pSTAT5 response from only core marker information.
I found that using methods like \gls{RF} and \gls{MARS},
I could approximate the pSTAT5 response with a multivariate non-linear function within
the CD4\positive lymphocytes.
%This could be useful for predicting the pSTAT5 response from only core marker information.
%and contain more than N points.
%Need to give more weight to pSTAT5 in clustering.  
%It has been hypothesised that certain lymphocyte subsets are less responsive to \cytokine{IL-2} stimulation in type 1 diabetics than in healthy controls.  
%How many cell pops are they?
\gls{RF} is ran with the default parameters which grows 500 trees up to x nodes.
Each node in the tree is one the core markers which is used to split the data.
Two important summary statistics of a RF are the \gls{RSS} and the IncNodePurity parameter, a measure of variable importance in the
construction of the random forests, which are averaged over all trees in the forest.
At a node the IncNodePurity is defined as the expected \gls{RSS} decrease before and afer the split.
The IncNodePurity parameter of the RF is average overall all nodes in over all trees it the forest.
The average variance explained was \pct{40} over all \gls{RF} with some \gls{RF} explaining between \pct{20} and \pct{55} of the variance.
In \Cref{figure:IncNodePurity}, the IncNodePurity is normalised per forest so that it lies between 0 and 1, to make it comparable across forests.
Interestingly, the CD45RA marker is more important at explaining the variation in pSTAT5 response than CD25.

However, extracting a gating strategy from these models is not obvious.
Here we seek an interpretable model which can be summarised as rules to identify groups of cells which are sensitive to proleukin.
Ideally, we would like to identify cells types which are unimodal on pSTAT5 (like the naive and memory T regs) through proleukin doses,
although these cells may not necessarily be unimodal on the other markers.
%Also a RF applied on sample is not generalisable to another sample as it sensitive to small changes in the data, though less so than a CART.  
\myfigure{scale=.5}
{IncNodePurity}
{
  Distribution across all samples of the normalised IncNodePurity parameter from \acrfull{RF}
}
{
  The IncNodePurity parameter is a measure of variable importance in the construction of the \acrfull{RF}.
  Interestingly, the CD45RA marker is more important than the CD25 marker in the pSTAT5 response prediction.
  This could because of the distinctly bimodal nature of the CD45RA distribution, which allows discrimination of memory cells which are high in CD25
  from naive cells, which are lower in CD25.
}


