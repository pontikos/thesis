
\subsection{pSTAT5 response cell phenotypes}

The sensitivity of the response in the four manually gated cell subsets is confirmed by plotting proleukin dose against pSTAT5 MFI (\Cref{figure:pstat5-mfi-auc-celltypes}.a).
It also illustrates that pSTAT5 saturation occurs in naive and memory regulatory T cells at 10 units of proleukin while naive effector cells do not reach
saturation even at the highest dose of 1000 units.
%Provided the pSTAT5 distribution is unimodal then a shift in the peak of the pSTAT5 density should correspond to a difference in the MFI,
%however this does not hold if the distribution is not unimodal as seen for the naive effectors (\Cref{figure:dose-effect-pstat5-cellsubsets}).
An alternative metric which captures the response of the whole distribution across all doses instead of single dose, 
is the area between the pSTAT5 cumulative density functions in the resting and stimulated samples.
This relates to the Kolmogorov-Smirnoff distance which is defined as the maximum vertical distance between cumulative density functions,
but also captures the horizontal dimension.
%The repeatability of these metrics was tested but I found none to be repeatable neither in the gated subsets nor any of the larger subsets.
%(\Cref{XXX}).
%Instead I decided to use the area under the pSTAT5 dose-response curve for each cell type (\Cref{figure:pstat5-auc-celltypes}).

%The repeatability of pSTAT5 MFI at all doses in each cell subset was tested using the manual gating and my method which allowed for
%the gates to move so as to recenter on the data (\Cref{figure:repeatability-gating}).
%It was found that overall the pSTAT5 MFI was poorly repeatable using both gating approaches,
%with the most sensitive cells to proleukin, showing the worse repeatabilty in pSTAT5 MFI.
%To capture the response across all days, the sum of the pSTAT5 MFI was used.
%and not just the portion which crosses the positivity treshold,
%There are two established metrics for measuring the distance between distributions:
%the Kullback-Leibler (KL) and Kolomogorov-Smirnoff (KS) distance.
%the KL distance is not symmetric

%The relative shift in the pSTAT5 distribution between resting and stimulated can be measured by computing the area between the cumulative density functions of the unstimulated and stimulated samples.
% As the pSTAT5 signal increases in a stimulated sample it is expected that the CDF shifts right.
% The resting provides the baseline.
%The area between the pSTAT5 CDF in the resting sample and the pSTAT5 CDF in the stimulated sample is measure of the strength of the response.
%Ten of these low responders from the Cambridge Bioresource were recalled for re-analysis to assess stability of this phenotype.
%However this cell phenotype was not found to be reproducible.  
%Using the automatically adjusted gates from manual gating, we assess the repeatability of the response in the four identified cell subsets.
%Defining a threshold for pSTAT5 positivity is done using a similar approach as in \Cref{chapter:il2ra}.
%The top percentile of the pSTAT5 distribution in the resting sample is used as a threshold as described in %\Cref{ }.  
%Some individuals with a reduced pSTAT5 response consistent across studied cell types were identified by Tony, however there was no evidence of association with disease status.  
%There are unsolved day to day variation issues, better to do in bulk from frozen once Foxp3 IC stain works
%A question of potential interest with whether the low responders show any other immunological phenotype such as for example Th17, IL-21 production, Tfh proportion?

%Normalisation can rely on peak alignment of the ungated sample but the peaks are often hard to reliably identify.
%The fluorescent marker required for pSTAT5 is Alexa Fluor 488
%Absolute MFI is a not a reliable way to assess pSTAT5.
%We need to subtract the background mfi.
%We can either do this on a cluster basis or use ann joining.
%What is more repeatable?  


