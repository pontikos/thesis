% ************************** Thesis Acknowledgements *****************************

\begin{acknowledgements}      


 I am very grateful for having been offered the great opportunity to undertake this PhD and to learn so much about statistics, genetics, and science in general.
 I am certain that these three years of intellectual development have truly reshaped the way I think and see the world and will undoubtedly bear great positive influence on my future life.
 %This has been an intellectual adventure which has in the space of three years completely reshaped 

%And lastly my eternal gratitude to my Mother for instilling in me the love of science.
 First and foremost, my eternal gratitude to my Mother for having devoted so much time to my upbringing and to have kindled my enthusiasm for science and adventure from a very young age.
 Although she is sadly no longer among us, she is the reason why I decided to dedicate my computer science and programming skills to scientific and medical research.

 Secondly, I wish to thank all the great people I have had the privilege to know and work with during this academic adventure.
 %Firstly for trusting my capabilities,
 In particular, my super-hero supervisor, Chris Wallace, for her efficiency, advice, support, and encouragement when I was feeling down, and her pragmatism, guidance, and helping me channel my enthusiasm productively when it was overflowing.
 %I am continuously amazed at how 
%It was wonderful having an advisor who had spent many years thinking deeply about the business of modeling, who also allowed me the freedom to make my own mistakes.
 I'd also like to thank my advisor, Anna Petrunkina Harrison, for providing much encouragement and support, and putting me in touch with the right people in flow cytometry.
 John Todd's intense presence and input have been incredibly valuable and I would like to thank him for his relentless drive, insight, experience and pushing me to publish.
 I wish to also acknowledge the Medical Research Council, Doctoral Training Grant (C006148), for funding my research.

 Having no formal training in statistics, I was privileged enough to be surrounded by smart people throughout my PhD from whom I have learned a great deal.
 Notably, Xin Yang, Hui Guo, Nick Cooper, Marina Evangelou, Mary Fortune and Arcadio Rubio Garcia spring to mind, with whom I have had many interesting lunchtime conversations.
 Mary Fortune's  mathematical prowess is only rivalled by her baking skills.
 Xin Yang also makes delicious chinese buns, a skill that will no doubt continue to serve her well.

 I also learned a much about biology thanks to discussions with Marcin Pekalski, Tony Cutler, Xaquin Dopico Castro, Ricardo Fereirra and Charlie Bell.
 They were also very important in making me feel welcome to the lab (perhaps even a little too welcome at times).
 I wish them all the best in their future endeavours and I am sure science or some other mysterious force will bring us back together.

 The first chapter, I owe in great part to my second supervisor Anna Petunkina-Harrison for the introduction to flow cytometry and putting me in touch with the flow cytometry community.
 I should also mention that my first year examiner, Lorenz Wernisch, gave me some very constructive feedback on my first year report and food for thought.

 The second chapter of my thesis would not have been possible without the previous work and extensive email communications with Vincent Plagnol, Calliope Dendrou and Linda Wicker.
 Notably, I would like to give special thanks to Vincent for offering me the opportunity of continuing to do what I love at University College London for the next two years.
 I am sure we will do some great work together.

 The third chapter was enabled by Tony Cutler's data set, but also benefitted greatly from Marcin Pekalski's and Charlie Bell's detailed knowledge of immune cell types and flow cytometric artefacts.

 The fourth chapter, which lead to a publication in BMC Genomics, was a continuation of the important work done by Helen Schuilenberg, Debbie Smyth, James Traherne and Jyothi Jayaraman, among others, on characterising KIR copy number variation.
 This body of work is also proof to the testimony of the kind of great work resulting from between lab collaborations.
 The success of this project rested primarily on the supervision of Chris Wallace, to which I am very grateful for the time and effort invested in the publication.
 I also received very meticulous feedback from John Todd and Jo Howson which greatly facilitated the submission procees.

 The fifth and final chapter, was greatly aided by discussions and proof-reading from Chris Wallace and Mary Fortune.

 However, no PhD would be complete without a little bit of extra-curricular adventures which surely should be the object of another thesis.
 My thanks to my team mates Sriyal Mendis and Gary Low for helping me out on my autorickshaw Indian chapter.
 Their driving skills kept me alive and unscathed during our 4000km autorickshaw adventure across India, in spite of my unfortunate cowllision in Ongole.

 Thanks to Ben Silva-Weatherley, Steve Sawiak, and the rest of the team at RNA Crossfit for making sure I didn't completely neglect my health through the daily PhD grind.
 I'd like to think I tried to stay true to the humanist ideal of a healthy mind in a healthy body ``mens sana in corpore sans''.

 I would also like thank to all the friends, both professionals and students, I have made during these five fun years at Cambridge, first at the European Bioinformatics Institute, then at the University and Addenbrookes.
 Notably, my good friends, Shawn O'Donell and Pallawi Sinha, who have often fed me and generally been very supportive and patient with my poor time-keeping (something I am still working on improving).
 Thanks also to all housemates of 386 Cherry Hinton Road, past and present, my home of five years.
 Thank you Cambridge for being such a friendly, unique, quircky and curious place, one which I have grown very fond of.

 Lastly and most importantly, my everlasting gratitude to my Father and my Brother, inspirations in their own right, for their unconditional love, incessant support and for listening.

\end{acknowledgements}
