\begin{abstract}

  Association testing is a crucial scientific tool in correlative biology to understand how genes and environment
  influence health and disease.
  In the course of the last 50 years, genetic association studies have discovered gene variants which significantly increase
  or decrease risk of disease, but also variants which correlate with finer quantitative phenotypes such as cell counts.
  These phenotypes promise to elucidate the molecular mechanisms which link genes, environment and phenotype.
  Hopefully this information will provide us a better understanding of how to intervene.
  An important factor when testing for association, is the accuracy with which a phenotype can be measured.
  Reproducibility is a major hurdle in biology,
  since small changes in experimental conditions imply that no experiment yields exactly the same result twice.
  Assessing the level experimental noise can be done thanks to repeated measures.
  %In particular, phenotypes more stable over time are more likely to show significant correlation with a genetic variant.
  Because of experimental noise generally large datasets often generated by high-throughput technologies, 
  are necessary to see patterns emerging.
  These large datasets require computational methods to analyse.
  These methods need to on one hand deal with noise is sensible method.
  Normalisation methods are used to make these datasets comparable.
  Firstly these datasets may not be directly comparable.
  Clustering is used to group data in meaningful ways.

  One such dataset which at the centre of this thesis is flow cytometry.
  In flow cytometry samples are taken at different times, prepared and analysed on different days.
  Normalisation is important 
  I cover several ways this can be achieved.
  The objective of flow cytometry data analysis is to identify different cell types.
  Manually analysed data suffers from operator bias.
  Another dataset in this thesis is qPCR which is applied here to determining copy numbers.
  Different plates have 
  I will also look at correlating qPCR data with SNP data using supervised clustering.
  Finally I conclude with what I have learned from applying these methods and how these
  could be developed further.




Since T1D is an autoimmune disease, there are quantitative features of the immune system, such as presence of insulin autoantibodies, which can predict its onset.
%A more promising form of screening which further elucidate the molecular mechanisms which link the genetic factors to the T1D phenotype, is to study the immune phenotype.
But higher resolution study of the immune phenotypes, involving quantification of the populations and types of cells of the immune system,
A tool to do this is flow cytometry.



However flow cytometry data analysis is hard.


\end{abstract}
