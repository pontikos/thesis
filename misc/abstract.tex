\begin{abstract}

Because T1D is an auto-immune disease, there are quantitative features of the immune system, such as presence of insulin auto-antibodies, which can predict its onset.
%A more promising form of screening which further elucidate the molecular mechanisms which link the genetic factors to the T1D phenotype, is to study the immune phenotype.
But higher resolution study of the immune phenotypes, involving quantification of the populations and types of cells of the immune system,
promises to elucidate the molecular mechanisms of T1D which link genes, environment and phenotype.
Hopefully this information will provide us a better understanding of how to intervene.
A tool to do this is flow cytometry.

In detecting genetic association, the accuracy at which we can measure a phenotype is an important factor in the strength of the association.
In particular more stable, lower variance, phenotypes are more likely to show significant correlation with a genetic variant.
The stability of a phenotype can be estimated from repeated measures.



However flow cytometry data analysis is hard.


  Reproducibility is a major hurdle in science an especially in biology.
  Manually analysed data suffers from operator bias.
  Also given the large datasets which are generated by high-throughput technologies, computational methods are compulsory.
  These methods need to on one hand deal with noise is sensible method.
\end{abstract}
